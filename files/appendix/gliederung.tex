\thispagestyle{empty}
\progressbarCalcBarSubDivision{3em}{5}
\newcounter{countprozent}
\newcommand{\printprozent}[1]{
  \ifthenelse{#1 < 10}
    {\progressbar{0.0#1}}
    {\progressbar{0.#1}} & #1
    \addtocounter{countprozent}{#1}
}
\renewcommand{\longtableheader}{\hline \multicolumn{1}{|c}{\textbf{ID}} & \multicolumn{1}{c}{\textbf{Arbeitstitel}}
& \multicolumn{2}{c}{\textbf{Status}} & \multicolumn{1}{c|}{\textbf{Bemerkung}} \\
& & \multicolumn{1}{c}{bildlich} & \multicolumn{1}{c}{\%} & \\ \hline}
\newcommand{\longtableHeader}{
\longtableheader
\endfirsthead

\longtableheader
\endhead

\hline \multicolumn{5}{|r|}{\longtableendfoot} \\ \hline
\endfoot

\hline% \hline
\endlastfoot}

\newcommand{\subsecnameTeilone}{Bau des \TITEL}
\newcommand{\subsecnameTeiltwo}{Erweiterung um \acs{GPS} und Kompass}
\newcommand{\subsecnameTeilthree}{Erweiterung um Autonomes Reagieren (Hindernissen ausweichen)}
%%%%%%%%%%%
\section*{Gliederung}
\pdfbookmark[1]{Gliederung}{Gliederung}
\subsection*{Theoretischer Teil}
\pdfbookmark[2]{Theoretischer Teil}{TheoretischerTeil}


\resetplusplus
\begin{longtable}{|lllrl|}
\longtableHeader

\plusplus & Vorwort & \printprozent{0} & \\
\plusplus & \nameref{sec:einleitung} & \printprozent{95} & \siehe{sec:einleitung} \\
\plusplus & \nameref{sec:drohnen} ~-- \nameref{subsec:drohnen:begriffskläaerung} & \printprozent{90} &
\siehe{subsec:drohnen:begriffskläaerung} \\
\plusplus & \nameref{subsec:drohnen:AbwaegungderArgumente} & \printprozent{95} &
\siehe{subsec:drohnen:AbwaegungderArgumente} \\
\plusplus & \nameref{subsec:bau:bauteile} & \printprozent{86} &
\siehe{subsec:bau:bauteile} \\
\plusplus & \subsecnameTeilone & \printprozent{90} & \siehe{subsec:bau:Mikrokopter:teil1} \\
\plusplus & \subsecnameTeiltwo & \printprozent{90} & \siehe{subsec:bau:Mikrokopter:teil2} \\
\plusplus & \subsecnameTeilthree & \printprozent{15} & \siehe{sec:autonomes_fliegen} \\
\plusplus & Nachwort & \printprozent{0} & \\
%% \titleref{sec:drohnen}
\end{longtable}

\subsection*{Praktischer Teil}
\pdfbookmark[2]{Praktischer Teil}{PraktischerTeil}

\begin{longtable}{|lllrl|}
\longtableHeader

\plusplus & \subsecnameTeilone & \printprozent{98} & \\
\plusplus & \subsecnameTeiltwo & \printprozent{90} & \\
\plusplus & \subsecnameTeilthree & \printprozent{20} & \\
\end{longtable}

\newpage
\thispagestyle{empty}

\subsection*{Fortschritt insgesamt}
\pdfbookmark[2]{Fortschritt insgesamt}{Fortschrittinsgesamt}
\progressbarCalcBarSubDivision{26em}{10}

\FPeval\myprogressdruchschnittkomma{\thecountprozent/\theplusplus}
\FPeval\myprogressdruchschnittkomma{\myprogressdruchschnittkomma/100}
\FPround\myprogressdruchschnittkomma\myprogressdruchschnittkomma3
\FPeval\myprogressdruchschnitt{\myprogressdruchschnittkomma*1000}
\FPtrunc\myprogressdruchschnitt\myprogressdruchschnitt0
\Calcprogresspromille

\begin{longtable}{|lll|}
\hline \multicolumn{1}{|c}{\textbf{Bezeichnung}} & \multicolumn{1}{c}{\textbf{Fortschrittsbalken}} &
\multicolumn{1}{c|}{\textbf{\textperthousand}} \\ \hline
\endfirsthead

\hline \multicolumn{1}{|c}{\textbf{Bezeichnung}} & \multicolumn{1}{c}{\textbf{Fortschrittsbalken}} &
\multicolumn{1}{c|}{\textbf{\textperthousand}} \\ \hline
\endhead

\hline \multicolumn{3}{|r|}{\longtableendfoot} \\ \hline
\endfoot

\hline% \hline
\endlastfoot

Durchschnitt\footnote{Der Durchschnitt aus den oben angegebenen \arabic{plusplus} Prozentwerten, es ist keine Wertung des
Umfangs mit in die Berechnung eingeflossen, was diese eventuell verfälscht\dots} &
\progressbar{\myprogressdruchschnittkomma} &
\myprogressdruchschnitt \\
Eigenes Ermessen & \progressbar{.\FortschritteigenesErmessen} & \FortschritteigenesErmessen \\
Zeitfortschritt\footnote{Der Zeitfortschritt wurde über die schon abgelaufene Zeit berechnet} &
\progressbar{\myprogresskomma} &
\myprogress \\
\end{longtable}