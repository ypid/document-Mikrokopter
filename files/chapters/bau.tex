\section{Der Bau des Mikrokopter}
\label{sec:bau:Mikrokopter}

\subsection{Einleitung}
\label{subsec:bau:Mikrokopter:Einleitung}

Der Bau verlief in mehreren Phasen.
Zuerst, da ich anfangs fliegen üben wollte
und um später die Funktionalität des Mikrokopters nach und nach zu erweitern.
Um bei möglichen Abstürzen nicht sinnlos viel Hardware ersetzen zu müssen.
Im folgenden ist eine Erklärung der einzelnen Hardware Komponenten zu finden.
Eine \nameref{anhang:HardwareUndPreisliste} ist im Anhang \vpageref{anhang:HardwareUndPreisliste}.

Anschließend beschreibe ich im ersten Teil den Aufbau des Mikrokopters
und im folgenden Teil die Erweiterung um zusätzliche Sensoren
zur Erkennung der Lage im Raum.
Das Hauptthema meiner Projektarbeit folgt im nächsten Kapitel
(\nameref{sec:autonomes_fliegen})
ab Seite \pageref{sec:autonomes_fliegen}.
%die Fähigkeit des Autonomen reagieren auf die Umwelt.

Ich baue bei dieser Projektarbeit auf ein vorhandenes Selbstbauprojekt auf,
dem \enquote{Mikrokopter}.

\subsection{Bauteile}
\label{subsec:bau:bauteile}

\subsubsection{Rahmen}
%\begin{minipage}{8cm}
\begin{wrapfigure}{r}{0pt}
	\includegraphics[width=8cm]{files/images/Rahmen/1-rm-bg\imageresize}
	\captionof{figure}{Mikrokopter Rahmen (MK40)}
	\label{fig:MK40}
\end{wrapfigure}
%\end{minipage}

%\begin{minipage}{8.4cm}
Der Rahmen (MK40) bildet das Grundgerüst des Mikrokopter.
Es besteht aus zwei quadratischen Zentralplatten aus \ac{GFK}
und vier Auslegern aus Aluminium Vierkantrohre (\unit{10 x 10}{\milli\metre}).
Ein Ausleger ist rot lackiert und bildet den ersten Ausleger.
Die anderen drei Ausleger sind dagegen schwarz lackiert.
Der rote Ausleger bildet somit den ersten, vorderen Ausleger.
An diesem Orientieren sich die Steuerbefehle des Piloten.

In den Zentralplatten und den Auslegern sind die
Löcher für die Schrauben und Motoren bereits vorgebohrt.
Nach dem Zusammenbau beträgt
der Abstand zweier gegenüberliegender Motoren \unit{40}{\centi\metre}
und das Gewicht des Rahmens wie er auf Abb.~\vref{fig:MK40}
zu sehen ist circa \unit{105}{\gram}.
%\end{minipage}
%\hfill

In der Mitte des Kreuzes wird der Akku und die Elektronik befestigt.

\newpage

\subsubsection{Akku}
\begin{wrapfigure}{r}{0pt}
	\includegraphics[width=8cm]{files/images/Akku/IMG_7099-rm-bg\imageresize}
	\captionof{figure}{Lithium-Polymer-Akku}
\end{wrapfigure}

Der verwendete Akku ist ein im Modellbau
üblicher \ac{LiPo} Akku.
Ein \ac{LiPo} Akku gegenüber \ac{NiCd} oder \ac{NiMH}
hat den Vorteil der höheren Energiedichte
und einer schnelleren Energieauf-
beziehungsweise abgabe.

Die verwendeten Akkus haben, bei 3 in Reihe geschalteten Zellen,
eine Spannung von \unit{11{,}1}{\volt} und eine Kapazität von \unit{2200}{\milli\ampere\hour}.
Die Akkueinheit wiegt \unit{179}{\gram}.
Daraus ergibt sich eine Energiedichte von \unit{136{,}42}{\watt\hour\per\kilogram}.
Die Abmessungen betragen \unit{100 x 34 x 25}{\milli\metre}.

Der Akku wird mit der Verteilerplatine (in meinem Fall mit Klettband) verbunden.


\subsubsection{Verteilerplatine}
\begin{wrapfigure}{r}{0pt}
	\includegraphics[width=9cm]{files/images/Verteilerplatine/IMG_7084-rm-bg\imageresize}
	\captionof{figure}{Stromverteiler-Leiterkarte}
	\label{fig:Stromverteiler-Leiterkarte}
\end{wrapfigure}

Die Verteilerplatine\footnote{Stromverteiler-Leiterkarte (4fach-BL)}
verteilt den Strom vom Akku
und die Steuersignale von der Hauptplatine.
Somit sinkt die Anzahl der benötigten Kabel deutlich.
Auf dieser Verteilerplatine können vier Motorregler eingelötet
und dadurch mit Strom und der \ac{I2C} Schnittstelle verbunden werden.
Auf Abb.~\vref{fig:Stromverteiler-Leiterkarte} sind
die vier Motorregler bereits auf die Verteilerplatine gelötet.
Es ist auch eine Transistorschaltung vorgesehen,
um zwei Verbraucher mit maximal \unit{800}{\milli\ampere}
pro Verbraucher schalten zu können.

Außerdem wird die komplette Stromversorgung des Mikrokopters über einen
elektronischen Leistungsschalter (BTS555), ein Halbleiterrelais, geschaltet.
Da die meisten erhältlichen mechanischen Kippschalter keine hohen Gleichströme
zuverlässig schalten können.
Weil bei Gleichstrom, anders als bei Wechselstrom,
ein viel längerer und anhaltender Lichtbogen entsteht,
der die Kontaktflächen des Schalters schnell abbrennen lässt.
Dadurch steigt sein Innenwiderstand.
Die folge: Hitzeentwicklung und Energieverlust.

Die Verteilerplatine versorgt die Hauptplatine mit Strom
und ist unterhalb der Hauptplatine am Rahmen angebracht.


\subsubsection{Hauptplatine (FlightCtrl)}
\begin{wrapfigure}{r}{0pt}
%\begin{figurewrapper}
	\includegraphics[width=7cm]{files/images/FC/oben-rm-bg\imageresize}
	\captionof{figure}{Hauptplatine (Flight-Ctrl V2.0 ME) von oben}
	\label{fig:FlightCtrl:oben}
%\end{figurewrapper}
\end{wrapfigure}

Die Hauptplatine, in Fachkreisen auch FlightCtrl genannt,
ist die zentrale Steuereinheit des Mikrokopters.
Diese ist zwingend notwendig um die Flugfähigkeit des Mikrokopters zu gewährleisten,
da mit dieser die komplette Lageregelung durchgeführt wird.

Die Hauptplatine hat die Abmessungen \unit{50 x 50}{\milli\metre}
und wiegt (bestückt) \unit{23}{\gram}.

Die Lageregelung geschieht mit drei Beschleunigungssensoren,
die je für eine Raumdimension messen können,
von wo die (Erd)gravitation wirkt.
%\fxnote{Ich möchte zum Ausdruck bringen das nicht nur die Erdgravitation,
%sondern eigentlich nur die stärkste Gravitation die Messergebnisse bestimmt}
Somit ist für die Berechnung die Position der Erde zum Mikrokopter bekannt.
Zusätzlich stehen noch drei
\href{http://de.wikipedia.org/wiki/Drehratensensor}{Drehratensensoren}
zur Erfassung der Drehgeschwindigkeit um die drei Achsen zur Verfügung.
Bewegt sich der Mikrokopter, dann wirkt die Gravitation
und die Eigenbeschleunigung auf die Sensoren
und somit ist die Position der Erde zum Mikrokopter nicht mehr allein
über die Beschleunigungssensoren ermittelbar.
Mit Hilfe dieser zwei Sensortypen und der Software, die vom
MikroKopter Projekt\footnote{Also hauptsächlich von Holger Buss und Ingo Busker} stammt,
kann sich der Mikrokopter eigenständig in einer waagerechten Position halten.
Die Hauptplatine ist hierfür mit den vier Motorreglern
über eine \ac{I2C} Schnittstelle verbunden.

Es steht auch ein barometrischer Luftdrucksensor zur Verfügung,
dessen Wert sich beim Einschalten des Mikrokopters auf die relative Höhe Null eingestellt.
Mit diesem Sensor kann der Mikrokopter seine Höhe selbst erfassen
und dadurch ist es möglich selbstständig eine vorgegebene Höhe halten.

Die dafür notwendige Berechnung werden mit Hilfe eines AVR Atmel Mikrocontroller durchgeführt.
Dieser ist auf Abb.~\vref{fig:FlightCtrl:oben} als schwarzes, quadratisches Halbleiterbauelement
mit vielen (44) Kontakten zu sehen.

Es stehen auch verschiedene Schnittstellen über Steckerleisten zur Verfügung.
Diese dienen dem Verbinden mit einem Computer
oder um weitere Module mit der Hauptplatine verbinden zu können.

Es werden auch direkt Schnittstellen für das Ansteuern von Servos bereitgestellt.
Servos sind im wesentlichen Elektromotoren samt ihrer Steuerelektronik
und Mechanik.
Diese können benutzt werden um, für eine Kamera, die Flugbewegungen
zu kompensieren.
Dadurch ist es möglich immer den gleichen Bildausschnitt aufzunehmen.
Was ohne aktive Kompensation nur mit Gewichten möglich wäre.
Diese Kompensation ist notwendig da sich der
Mikrokopter in die jeweilige Richtung beugen muss
um in diese Richtung zu fliegen.

Um als Pilot den Mikrokopter steuern zu können, ist die Hauptplatine mit einem,
im Modellbau üblichem, Funkempfänger verbunden.
Dieser empfängt die Steuerbefehle des Piloten über eine Antenne.

Die Befehle werden Ausgewertet und in Form von Drehzahlenwerten an die
Motorregler weitergegeben.


\subsubsection{Motorregler}
\begin{wrapfigure}{r}{0pt}
	\includegraphics[width=7cm]{files/images/BL-Ctrl/oben-rm-bg}
	\captionof{figure}{Motorregler (Brushless Regler V1.2)}
\end{wrapfigure}

Der Motorregler ist für die Ansteuerung eines Brushless-Motors zuständig.
Seine Abmessungen betragen \unit{20 x 43}{\milli\metre}.

Er bekommt die jeweiligen Sollwerte von der Hauptplatine
mit einer Frequenz von \unit{500}{\hertz}
und setzt sie in Steuerbefehle für die Motoren um.
Um dies zu erreichen, wird je ein Atmel AVR Mikrocontroller für einen Motor benutzt.

Da für einen Brushless-Motor das Drehfeld\footnote{Ein Drehfeld
ist ein Magnetfeld, das sich um eine Rotationsachse dreht.}
drei Pole mit einer hohen elektrischen Leistung\footnote{Der verwendete Motorregler
ist für eine Dauerbelastung von \SIrange{10}{12}{\ampere}
und eine kurzzeitige Spitzenbelastung von \unit{20}{\ampere} ausgelegt.}
geschaltet werden muss, werden für einen Brushless-Motor drei \ac{MOSFET} als Steuertransistor benutzt.

Die Drehzahlwerte werden in Form eines Drehfeldes an die Motoren weitergeben.

\newpage

\subsubsection{Brushless-Motor}
\begin{wrapfigure}{r}{0pt}
	\includegraphics[width=7cm]{files/images/Brushless-Mortor/Motor-rm-bg\imageresize}
	\captionof{figure}{Brushless-Motor der Firma Robbe}
	\label{fig:Brushless-Mortor-Robbe}
\end{wrapfigure}

Die Motoren, die sich im MikroKopter Projekt durchgesetzt haben sind bürstenlose Gleichstrommotoren.
Der wichtigste Unterschied zu einem \enquote{normalen} Elektromotor ist,
dass keine Bürsten (Schleifkontakte) benötigt werden,
da der bewegliche Teil des Motors mit Permanentmagneten ausgestattet ist
und auf dem unbeweglichem Teil das Drehfeld erzeugt wird.
Somit ergeben sich folgende Vorteile:
\begin{itemize}
	\item Hoher Wirkungsgrad (da keine Bürsten)
	\item Hohe Lebensdauer
		(wegen dem sehr geringen Verschleiß $\rightarrow$ keine Bürsten)
	\item Gute Regelbarkeit
		(Drehfeld wird extern erzeugt $\rightarrow$ kann gemessen werden)
\end{itemize}

Jeweils ein Brushless-Motor wird von einem Motorregler kontrolliert.
Diese sind über drei Pole miteinander verbunden.

In meiner Konfiguration kommen die Brushless-Motor von Robbe ROXXY
BL-Outrunner 2824-34 zum Einsatz.
% Abb.~\vref{fig:Brushless-Mortor-Robbe}
Diese haben die Abmessungen \unit{28,8 x 26}{\milli\metre}
und wiegen \unit{48}{\gram}.

Auf dem mit Permanentmagneten ausgestattete Teil des Motors (Glocke),
können die Propeller die Befehle der Hauptplatine in Auftrieb umsetzen.


\subsubsection{Propeller}
\begin{figurewrapper}
	\includegraphics[width=14cm]{files/images/Propeller/IMG_7114-rm-bg\imageresize}
	\captionof{figure}{Propellerpaar (je ein rechts- und linksdrehender)}
\end{figurewrapper}

Die verwendeten Propeller (MaxxProducts EPP0845) sind aus Kunststoff gefertigt.
Diese können mit Hilfe eines Mitnehmers an den Brushless-Motoren befestigt werden.

Der Durchmesser eines Propellers beträgt \unit{25,4}{\centi\metre}.
Es gibt je zwei rechts- und linksdrehenden Propeller.
Der Grund hierfür ist bereits im Kapitel
\Siehe{subsubsec:drohnen:AbwaegungderArgumente:Zivilen} geklärt.

\bigskip

Die Befehle für die Propeller, die Geschwindigkeit zu ändern, kommen
indirekt vom Piloten. Diese Befehle werden mit einem Funkempfänger
von der Hauptplatine empfangen.


\subsubsection{Modellbau Funkempfänger}
\begin{wrapfigure}{r}{0pt}
	\includegraphics[width=7cm]{files/images/Funkempfaenger/IMG_7119-rm-bg\imageresize}
	\captionof{figure}{Modellbau Funkempfänger}
	\label{fig:Modellbau-Funkempfaenger}
\end{wrapfigure}

Der Modellbau Funkempfänger (ACT DSL-4top) der zum Einsatz kommt,
arbeitet Unverschlüsselt und Analog auf einer Frequenz von \unit{35}{\mega\hertz}.

Das Gewicht beträgt circa \unit{6}{\gram}
bei den Maßen \unit{36 x 13 x 16}{\milli\metre}.

Um die Trägerfrequenz von etwa \unit{35}{\mega\hertz}
zu erzeugen kommt ein Quarz zum Einsatz,
von diesem ist rechts auf der Abb.~\vref{fig:Modellbau-Funkempfaenger}
das Metallgehäuse erkennbar.
Der Sender braucht die gleiche Frequenz um dem Empfänger Signale übermitteln zu könne.
Hierfür wird ebenfalls ein Quarz verwendet.

Die Hauptplatine wird an der ersten dreipoligen Stiftleiste angeschlossen,
da hier das Summensignal anliegt.
Über dieses Summensignal werden alle 9 Kanäle an die Hauptplatine übertragen.

Diese Informationsübertragung geht allerdings nur in eine Richtung (unidirektional).
Sollen Informationen vom Mikrokopter zurück an den Piloten übertragen werden,
kann auf Bluetooth zurückgegriffen werden.


\subsubsection{Bluetooth Funkmodul}

Das Bluetooth Funkmodul ermöglicht eine digitale drahtlose Kommunikation mit einem Computer
oder mit einem anderen Mikrokopter.

Das eigentliche Bluetooth Funkmodul (F2M03GXA) arbeitet mit \unit{3{,}3}{\volt}.
Da die Hauptplatine aber mit \unit{5}{\volt} Betriebsspannung arbeitet,
wird noch eine Adapterplatine (BT-AP10) benötigt.
Diese erlaubt den Betrieb dieses Bluetooth Funkmoduls.
Das Bluetooth Funkmodul hat sehr gute Funkübertragungseigenschaften
und ist sehr Energie effizient.
Es hat eine Reichweite bis zu \unit{300}{\meter},
wenn als Gegenstück das gleiche Bluetooth Funkmodul zum Einsatz kommt.
%was eigentlich in keiner Bluetooth Klasse spezifiziert ist
%und somit nur mit dem gleichen Bluetooth Modul als Gegenstelle erreicht werden kann.
Die Bluetooth Verbindung ist hierbei verschlüsselt.

\begin{wrapfigure}{r}{0pt}
\begin{minipage}{7cm}
	\centering
	Adapterplatine~~~~~~~ \\
	\includegraphics[width=7cm]{files/images/BT-AP10/BT-Modul\imageresize} \\
	\vspace{-0.2cm}
	Modul F2M03GXA~~~~
	\captionof{figure}{Bluetooth Funkmodul}
	\label{fig:Bluetooth-Funkmodul}
\end{minipage}
\end{wrapfigure}
Zusätzlich können Störungen auf einzelnen Kanälen erkannt
und diese als unbenutzbar markiert werden.

Auf Abb.~\vref{fig:Bluetooth-Funkmodul} ist in der oberen Hälfte
die selbst gelötete Adapterplatine mit dem auf der Rückseite angelötetem
Bluetooth Modul zu sehen.
Die untere Hälfte zeigt das Bluetooth Modul von der bestückten Seite.

Eine digitale Funkverbindung ist auch notwendig, um dem Mikrokopter im Flug
Zielkoordinaten zu schicken, sodass dies angeflogen werden können.
Die Navigation wird von dem Navigationsboard erledigt
an dem auch das Bluetooth Modul angeschlossen ist.


\subsubsection{Navigationsboard}
\begin{wrapfigure}{r}{0pt}
	\includegraphics[width=7cm]{files/images/Navi-Ctrl/oben-1-rm-bg\imageresize}
	\captionof{figure}{Navigationsboard (Navi-Ctrl)}
\end{wrapfigure}

Das Navigationsboard (Navi-Ctrl) ist die Navigationsrecheneinheit.
Diese wird benötigt, um die \ac{GPS}- und Kompassdaten zu verrechnen.
Diese Daten werden benötigt um zum Beispiel
Zielkoordinaten anzufliegen.

Die Abmessungen betragen \unit{50 x 50}{\milli\metre}.
Hauptbestandteil des Navigationsboard ist ein ARM9 Mikrocontroller,
ein micro \ac{SD} Kartenleser und mehrere Schnittstellen.

Ist eine Speicherkarte im Kartenleser, dann kann die Firmware des
Navigationsboard in einem gewissen Intervall (Standard: eine Sekunde)
Statusinformationen auf die Karte sichern.
Diese Statusinformationen umfassen beispielsweise die \ac{GPS}-Koordinaten,
die aus dem \ac{GPS}-Signal berechnete
\href{http://de.wikipedia.org/wiki/GPS-Zeit}{Uhrzeit},
die Lage des magnetischen Nordpols und die Akkuspannung.

Außerdem existieren noch mehrere Schnittstellen die zum Verbinden
des Navigationsboard zu der Hauptplatine, dem Magnetfeldsensor
und dem \acs{GPS}-Empfänger dienen.

Das Navigationsboard hat keinerlei Sensoren,
sondern ist lediglich eine notwendige Adapter- und Recheneinheit
um \acs{GPS}- und Kompassdaten auszuwerten.
Die Sensoren werden als Module angeschlossen.
Der Magnetfeldsensor ist direkt auf dem Navigationsboard befestigt.


\subsubsection{Magnetfeldsensor}

Der verwendete Magnetfeldsensor (MK3Mag) ist ein 3-Achsen-Magnetfeldsensor.
Somit kann das Erdmagnetfeld im dreidimensionalen Raum erfasst werden.

Die Abmessungen betragen \unit{38 x 22 x 10}{\milli\metre}.

Die Information der Lage des magnetischen Nordpols
kann dann dazu eingesetzt werden, um die Steuerbefehle des Piloten
von der Blickrichtung des Mikrokopters zu entkoppeln.
Das bedeutet das sich die Steuerbefehle nicht mehr auf die aktuelle Blickrichtung,
sondern auf die Blickrichtung, die beim Einschalten erfasst wurde, beziehen.
Somit muss der Pilot nicht mehr in Kenntnis von der aktuellen Blickrichtung
des Mikrokopters sein, um diesen zu steuern.
\begin{wrapfigure}{r}{0pt}
	\includegraphics[width=7cm]{files/images/MK3Mag/oben-1-rm-bg\imageresize}
	\captionof{figure}{Magnetfeldsensor (MK3Mag)}
\end{wrapfigure}
Wenn man beispielsweise das Kommando \enquote{neige dich nach \enquote{Vorne}}
gibt, dann ist \enquote{Vorne} nicht mehr der rote Ausleger sondern
die Himmelsrichtung auf die der rote Ausleger beim Start der Motoren gezeigt hat.

Der Magnetfeldsensor ist auch erforderlich, um eine zuverlässige
Navigation über \ac{GPS} zu ermöglichen.
Da man über \ac{GPS} nur einen Punkt im
dreidimensionalen Raum\footnote{genau genommen in der vierdimensionalen Raumzeit}
erhält.
Daraus geht aber noch nicht die Raum Orientierung des Mikrokopters hervor.
Die Richtung, in die der rote Ausleger zeigt, lässt sich über \ac{GPS} nur erfahren,
wenn sich der Mikrokopter bewegt.


\subsubsection{\acs{GPS}-Empfänger}

Der \ac{GPS}-Empfänger (MKGPS) ist für den Empfänger und die Auswertung des \ac{GPS}-Signals zuständig.
Die Informationen, die sich aus diesem ergeben, werden, dann an das Navigationsboard weitergegeben.
\newpage

\begin{wrapfigure}{r}{0pt}
	\includegraphics[width=6cm]{files/images/MKGPS/unten-rm-bg\imageresize}
	\captionof{figure}{\acs{GPS}-Empfänger (MKGPS)}
\end{wrapfigure}
Die Platine arbeitet im Normalfall mit einer passiven Antenne.
Passive Antenne bedeutet, dass diese nicht über einen eingebauten Signalverstärker verfügt.
Zur Erhöhung der Genauigkeit kann auch eine aktive Antenne benutzt werden,
die aber auch teurer sind.
Die \ac{GPS}-Auflösung die rechnerisch erfasst werden kann,
beträgt \unit{1}{\centi\metre}.
Dieser Wert wird allerdings nicht erreicht,
da die Genauigkeit des, für das US-Militär entwickelten \ac{GPS},
für Zivilisten nicht so hoch ist.
Der Grund dafür ist, dass die Korrekturinformationen von Referenzstationen nicht frei Verfügbar
sind.

Um schneller die Position des Mikrokopters ermitteln zu können,
ist außerdem noch eine wiederaufladbare Knopfzelle auf der Platine aufgelötet.
Diese ermöglicht es den Almanach\footnote{Daten zu den Umlaufbahnen aller \ac{GPS}-Satelliten.}
und die aktuelle Zeit zu halten.
Dies wird getan weil die Datenrate über \ac{GPS} sehr langsam ist (\unit{50}{Bit\per\second})
und deswegen die Übertragung des kompletten Almanachs 12,5 Minuten dauern würde.
%\wrapfill

\acused{USB}
Außerdem befindet sich auf der Platine eine Lötstelle für einen \ac{USB}-Anschluss. Der
\ac{USB}-Anschluss ist direkt mit der \ac{USB}-Schnittstelle vom Empfänger-Modul\footnote{LEA-4H
von \href{http://www.u-blox.com/}{u-blox}.} verbunden.
Dadurch kann man den \ac{GPS}-Empfänger auch vom Computer aus leicht abfragen.



\subsection{Teil 1: Grundsätzlicher Aufbau}
\label{subsec:bau:Mikrokopter:teil1}

Zuerst montierte ich den Rahmen, der aus vier Auslegern aus Aluminium
und zwei quadratischen Zentralplatten aus \ac{GFK} bestand,
mit Schrauben und befestigte das Landegestell.
An dem Landegestell befestigte ich mit acht halbierten Listerklemmen
zwei \unit{31}{\centi\metre} lange und \unit{3}{\milli\metre} starke Eisenstangen.
Um zu vermeiden, dass sich die Schrauben
von selbst lösen verwendete ich bei allen Schrauben einen Sicherheitslack mittlerer Härte.
Am Rahmen, zwischen dem Landegestell, brachte ich Klettverschlüsse als Halterung für den Akku an.

Jetzt folgt die Elektronik.
Dazu gehört die Hauptplatine,
die bereits mit allen \acs{SMD} Teile bestückt war.
\acsu{SMD} bedeutet surface-mounted device: oberflächenmontiertes Bauelement.
Die Größe der \ac{SMD}s liegt in der Regel
im Millimeterbereich und sie sind daher relativ schwer, für eine Person, zu löten.

%% Bild SMD-Transistor
%\begin{wrapfigure}{r}{0pt}
\begin{figurewrapper}
%\begin{figure}\centering
	\includegraphics[width=7cm]{files/images/SMD-Transistor/2-rm-bg-cut}\hspace{5cm}
	% \imageresize ist zu schlecht
	\captionof{figure}{Ein SMD-Transistor}
	\label{fig:SMD-transistor}
%\end{figure}
\end{figurewrapper}
%\end{wrapfigure}

Es waren allerdings auch noch einige größere Bauteile zu bestücken.
Zum Beispiel ein Luftdrucksensor, Steckerleisten, Kondensatoren und diverse Kabel.

Dann musste bei jedem der vier Motorregler ein Kondensator
zum Glätten von Spannungsspitzen eingelötet
und anschließend eine Lötbrücke zur Adressierung der vier Motorregler gesetzt werden.
Um alle Motoren unabhängig voneinander über die gleiche \ac{I2C} Schnittstelle anzusteuern zu können.

Die vier Motorregler lötete ich dann an eine Verteilerplatine,
die unter anderem den Strom und die Steuerbefehle von der
Hauptplatine an die Motoren verteilt.

Leiterbahnen auf Platinen sind auf Grund ihres geringen Querschnittes
nicht geeignet für den Transport hoher\footnote{circa \unit{60}{\ampere} beim Einschalten}
Stromstärken.
Deshalb lötete ich zusätzlichen Kupferdraht auf die stark belasteten Stellen.
Das sind die Leiterbahnen die vom Akku zu den Motorreglern führen.
Die Länge der so verstärkten Leiterbahnen betrug circa \unit{18}{\centi\meter}.
Auf die Verteilerplatine lötete ich noch einen elektronischen Leistungsschalter,
zum Unterbrechen der kompletten Stromversorgung des Mikrokopters ein.

Ich lötete noch jeweils zwei \acs{SMD}-Widerstände
und \acs{SMD}-Transistoren auf die Verteilerplatine,
damit zwei Verbraucher über die Steuerausgänge der Hauptplatine geschaltet werden können.
Als Verbraucher lötete ich zwei \ac{LED}-Streifen ein.
Sie sollen den vorderen und hinteren Ausleger markieren.

Um die Lötstellen mechanisch zu entlasten,
verklebte ich die entscheidenden Stellen zwischen den Motorreglern
und der Verteilerplatine mit Heißkleber.

Später befestigte ich die Verteilerplatine am Rahmen.
An diese Verteilerplatine lötete ich dann noch einen Stecker zum Anschluss des Akkus an
und an die Motorregler jeweils drei Kabel, die mit Steckverbindungen zu je einem Motor führen.
Ich lötete noch die entsprechenden Steckergegenstücke an die Kabel der Motoren,
schraubte einen Mitnehmer für die Propeller an die Motoren und
die Motoren an den Rahmen.
Die Stecker wurden jeweils mit Schrumpfschläuchen gegen Kurzschlüsse isoliert.
Als nächstes befestigte ich die Hauptplatine mit schwingungsdämpfenden Abstandshaltern auf der
zweiten Ebene auf der Verteilerplatine und verkabelte diese miteinander.

Dann begann ich einige \acs{SMD}-Bauteile auf eine Adapterplatine zu löten,
die die Verbindung zwischen einem Bluetooth Funkmodul und der Hauptplatine ermöglicht.
Um Kurzschlüsse zu vermeiden, bekam dieses Bluetooth Modul eine Schrumpfschlauchhülle.
Das Modul habe ich dann mit der Hauptplatine verbunden und im Rahmen untergebracht.

Dann brachte ich am äußeren Ende des hinteren Auslegers einen Summer an,
den ich ebenfalls an die Hauptplatine anschloss.
Dieser Summer dient dazu, dem Piloten während des Fluges,
akustisch Fehler mitzuteilen oder Befehle zu bestätigen.
Ein Befehl der akustisch bestätigen wird ist
beispielsweise das Kalibrieren des Magnetfeldsensors
oder das erreichen eines \ac{GPS}-Wegpunktes.

Es folgte noch ein Modellbau Empfänger,
den ich am Rahmen befestigte, an die Hauptplatine anschloss und
die Antenne mit einem kleinen Röhrchen in die Vertikale brachte.

Als Abschluss für diesen zweistöckigen Turm (Verteilerplatine und Hauptplatine),
befestigte ich noch eine quadratische Plexiglasplatte.
Zum Schluss fixierte ich noch alle Kabel mit Kabelbindern
und montierte noch die vier Propeller auf die Motoren.
Es gibt jeweils zwei rechtsläufige und zwei linksläufige Propeller.
Das Funktionsprinzip eines Quadrokopters ist bereits im Kapitel
\printfullref{subsubsec:drohnen:AbwaegungderArgumente:Zivilen} geklärt.

Damit war die erste Bauphase in circa 18 Stunden beendet.
Es folgte noch die Konfiguration der Software
und Kalibrieren der Sensoren.

Das Ergebnis ist in Abb.~\vref{fig:MikroKopter-nach-Bauphase1} zu sehen.
Allerdings ohne die vier Ausleger und Rotoren.

\begin{figure}\centering
	\includegraphics[width=14cm]{files/images/Mikrokopter/IMG_6832-rm-bg\imageresize}
	\captionof{figure}{Mikrokopter: Blick auf die Hauptplatine, Motorregler und Rahmen}
	\label{fig:MikroKopter-nach-Bauphase1}
\end{figure}

\subsection{Teil 2: Erweiterung um \acs{GPS} und Kompass}
\label{subsec:bau:Mikrokopter:teil2}

In der zweiten Bauphase beschreibe ich die Erweiterung um \ac{GPS}
und einen Magnetfeldsensor.
Dies Erweiterung ist dazu gedacht,
dem Mikrokopter eine Orientierung und Navigation im Raum zu ermöglichen.

Auf den Platinen waren bereits die \acs{SMD}-Bauteile vorbestückt.
Ich lötete nur noch einen Kondensator
und Steckleisten auf das Navigationsboard, das die Verbindung zwischen Hauptplatine
und Navigationserweiterungen herstellt.
Dann lötete ich eine Steckleiste auf den Magnetfeldsensor
und montierte diesen auf das Navigationsboard.
Anschließend schraubte ich dies auf die
Hauptplatine und verkabelte sie miteinander.
Nun kam noch der \ac{GPS}-Empfänger
mit einer passiven \ac{GPS} Antenne
und einer Lötbrücke, zur Auswahl der Empfangsart.
Hier war es besonders wichtig eine elektrostatische Entladung zu vermeiden,
da der \ac{GPS}-Empfänger sehr empfindlich ist
und dadurch leicht zerstört werden kann.
Elektrostatische Entladung ist ein Ausgleich zweier unterschiedlich starker Ladungen.
Ursache dieser Ladungsdifferenz ist meistens Aufladung durch Reibungselektrizität.
Kommt es zu einem Ladungsausgleich über einen Halbleiter, kann es,
bei entsprechend hohem Strom, zur Zerstörung von Halbleitern kommen.

\begin{figure}\centering
	\includegraphics[width=16cm]{files/images/Navi/1-rm-bg\imageresize}
	\captionof{figure}{Navigationsboard (NaviCtrl), Magnetfeldsensor (MK3Mag) und SMD-Empfänger (MKGPS)}
	\label{fig:Navigationseinheit}
\end{figure}

Danach montierte ich den \ac{GPS}-Empfänger auf das Navigationsboard.
Abschließend kam eine Plexiglasplatte zum Schutz über den \ac{GPS}-Empfänger.
Diese Bauphase dauerte circa 4 Stunden.

Somit stand mir eine flugfähige Plattform
für meine Weiteres vorgehen zur Verfügung.
