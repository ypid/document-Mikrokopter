\section{Autonomes fliegen}
\label{sec:autonomes_fliegen}

Um mein, in der \nameref{sec:einleitung} definiertes Ziel, zu erreichen,
musste jetzt noch eine Sensor- und Softwareeinheit gefunden werden.

Diese Einheit muss folgende Anforderungen erfüllen:
\begin{itemize}
	\item Sie muss mit der Hauptplatine kommunizieren können,
		um die Flugkorrekturen umzusetzen.
	\item Sie darf nicht zu Schwer sein (unter \unit{100}{\gram} ist vertretbar)
	\item Sie darf nicht zu viel Strom benötigen,
		da alles von einem Akku versorgt werden sollte
		(unter \unit{500}{\milli\ampere\hour} pro Flug ist vertretbar).
	\item Sie sollte möglichst früh ein Hindernis detektieren,
		um noch Zeit zum Ausweichen zu geben.
	\item Es sollte möglich sein, dieses System zu deaktivieren,
		um im Fehlerfall von Hardware oder Software meiner Erweiterung
		noch die Kontrolle zu ermöglichen.
	\item Die Kosten sollten \EUR{200}
		nicht überschreiten.
\end{itemize}

\subsection{Überlegung eines Ausweichalgorithmus}
Ein erster Ausweichalgorithmus, den ich mir überlegt habe, ist
in Abb. \vref{fig:Programmablauf} als Flussdiagramm abgebildet.
Ich plante einen Entfernungssensor der den
Abstand nach unten misst und zwei die möglichst weit versetzt am
vorderem Ausleger angebracht sind.

Das Ziel sei als \ac{GPS}-Koordinate bekannt.

Mit dem Entfernungssensor, der nach unten ausgerichtet ist,
sollte immer die gleiche Höhe über Grund gehalten werden.

Die zwei Entfernungssensoren, die am ersten Ausleger,
also in Flugrichtung angebracht sind,
sollten ein Hindernis erkennen.
Nach einer Prüfung ob rechts beziehungsweise links kein
Hindernis detektiert wurde sollte, solange
das Hindernis noch den Weiterflug verhindert,
nach rechts beziehungsweise links geflogen werden.
Also in die Richtung in der kein Hindernis detektiert wurde.
Wenn beide Sensoren das Hindernis melden, wird direkt
die Höhe um maximal \unit{10}{\metre} erhöht.
Falls das Hindernis in diesem Zustand des Algorithmus
immer noch den Weiterflug verhindert, sollte wieder zu
der Startkoordinate zurückgeflogen werden.

Die Häufigkeit der Entfernungsmessungen definierte ich mit
\unit{10}{\hertz} also 10 mal pro Sekunde.

\begin{figurewrapper}
        \includegraphics[width=13cm]{files/images/dia/Kollisionsvermeidung-Programmablauf-crop}
        \captionof{figure}{Programmablauf zur Kollisionsvermeidung}
        \label{fig:Programmablauf}
\end{figurewrapper}

Nach diesen Überlegungen begann ich mit der Suche nach passenden Sensoren.
Meine erste Idee war Ultraschall.

\subsection{Entfernungsmessung mit Ultraschall}
\subsubsection{Theorie}
Die Entfernungsmessung mit Ultraschall funktioniert
meist\footnote{bei dem Ultraschallsensor B214
kommt dagegen das Prinzip
der \enquote{akustischen Rückkopplung} zum Einsatz.
Dieses Prinzip lässt sich auch bei Mikrofon und Lautsprecher beobachten.}
nach dem Prinzip der Laufzeitmessung.
Das bedeutet, dass der Sensor, der aus einem Sender und Empfänger besteht,
Schallwellen erzeugt.
Diese breiten sich mit einer bekannten Geschwindigkeit (etwa \unit{330}{\metre\per\second}) aus.
Treffen die Schallwellen auf ein schallreflektierendes Objekt,
dann wird der Schall reflektiert.
Dieses Echo wird vom Sensor erfasst
und die Zeitdifferenz vom Aussenden bis zum Wiederkehren bestimmt.
Dieses Prinzip ist als Sonar bekannt.

Die Entfernung zu den Objekten, dessen Echos vom Sensor erfasst wurden,
lässt sich mit folgender Formel berechnen:

\begin{align}
	r &= \frac{c\cdot \Delta t}{2}
\end{align}

Bedeutung der Formelzeichen:
\begin{eqlist}
	\item[$r$] entspricht der Entfernung zu einem Objekt.
	\item[$c$] ist in dieser Formel nicht die Lichtgeschwindigkeit,
		sondern da es sich um Schallwellen handelt entspricht diese
		Konstante der Schallgeschwindigkeit in Luft
		(circa \unit{1235}{\kilo\metre\per\hour}).
	\item[$\Delta t$] entspricht der Signallaufzeit
		vom Aussenden bis zum Wiederkehren.
\end{eqlist}

\subsubsection{Praxis}
\begin{wrapfigure}{r}{0pt}
%\begin{figurewrapper}
	\includegraphics[width=6cm]{files/images/Ultraschall-Sensoren/IMG_7053-rm-bg\imageresize}
	\captionof{figure}{Ultraschallsensor (B214)}
%\end{figurewrapper}
\end{wrapfigure}
Ich begann, mit Ultraschall-Sensoren zu experimentieren.
Ich besorgte mir anfangs einen kleinen Bausatz\footnote{siehe Datenblatt des Bauteils:
\url{http://www.kemo-electronic.de/datasheets/b214.pdf}}
der eine angegebene Reichweite von 10--\unit{80}{\centi\metre} hat.
Ein entsprechend großes und schallreflektierendes Objekt was sich innerhalb
dieses Abstandes befindet wird über eine leuchtende \ac{LED}
signalisiert.
Die Empfindlichkeit lässt sich nachjustieren.

Leider war die Zuverlässigkeit dieses Bausatzes nicht sehr hoch.
Das bedeutet, dass zum Beispiel Wände nicht mehr erkannt wurden, wenn der Ultraschall-Sensor
auch nur um circa 20\textdegree\ vom rechten Winkel zur Wand abwich.
Außerdem war die maximal mögliche Entfernung,
aus der ein Objekt noch zuverlässig erkannt wurde,
mit \unit{20}{\centi\metre}, in praktischen Anwendungsbeispielen, zu gering.
Die Maße des Moduls sind mit \unit{55 x 45}{\milli\metre} auch etwas groß.

%\wrapfill
Damit fiel dieses Modul für meinen Anwendungsfall aus.

\bigskip
Nach längerer Suche stieß ich dann auf eine Ultraschall-Sensorreihe
(SRF der Firma Devantech),
die speziell für die Robotik entwickelt wurden.

Da ich mir zu dem Zeitpunkt noch nicht sicher wahr
wie ich einen Ultraschall-Sensor mit der Hauptplatine
des Mikrokopters verbinden kann,
beschaffte ich mir zwei Ultraschall-Sensoren
(SRF05\footnote{siehe Datenblatt des Bauteils:
\url{\urlSRFfive}} und
SRF08\footnote{siehe Datenblatt des Bauteils:
\url{\urlSRFeight}}).

\begin{wrapfigure}{r}{0pt}
%\begin{figurewrapper}
	\includegraphics[width=6cm]{files/images/Ultraschall-Sensoren/SRF08/IMG_7165-rm-bg\imageresize}
	\captionof{figure}{Ultraschallsensor (SRF08)}
%\end{figurewrapper}
\end{wrapfigure}

Der Ultraschall-Sensor SRF08 verfügt, im Gegensatz zum SRF05,
über einen \ac{I2C} Bus, denn ich dazu benutzen wollte,
um den Sensor direkt mit der Hauptplatine zu verbinden.
Dies hat den Vorteil das mir die Messdaten der Sensoren
direkt bei der Berechnung der Lagenregelung,
auf der Hauptplatine, zur Verfügung stehen.
Die \ac{I2C} Schnittstelle des SRF08 hat auch
den Vorteil das bis zu 16 dieser Sensoren von einem
\ac{I2C} Bus aus ansteuerbar sind.

Der SRF05 hingegen benötigt mindestens einen Port\footnote{Genauer gesagt einen digitalen Port, der
zwischen Eingabe und Ausgabe wechseln kann.}
über die komplette Zeit der Messung.
Außerdem stellt der SRF05 die Entfernung nicht in \si{\centi\metre} zur Verfügung,
sondern als ein High-Impuls, dessen Länge proportional zur Entfernung des Objektes ist.
Die gemessene Zeit muss durch 58 geteilt werden,
um die Entfernung in \si{\centi\metre} zu erhalten.

Der SRF08 hingegen führt die Messung selbstständig aus,
und kann, nach Beenden der Messung, über dessen Adresse
nach dem Ergebnis (unter anderem in \si{\centi\metre})
gefragt werden.
Somit hat der Mikrocontroller, von dem der Messbefehl ausging,
während der Messung Zeit für andere Operationen.

Die Messhäufigkeit der beiden Sensoren wird im wesentlichen
durch die Laufzeit der Ultraschallwellen bestimmt.
Das bedeutet, man kann in kürzeren Abständen messen,
wenn man nicht auf Echos von weiter entfernten Objekten wartet.
Die Messdauer beträgt bei \unit{6}{\metre} circa \unit{65}{\milli\second}.

In der folgenden Tabelle \vref{tab:Vergleich_SRF05_SRF05} sind weitere
Details der zwei Ultraschall-Sensoren gegenübergestellt.

\renewcommand{\longtableheader}{& \textbf{SRF05} & \textbf{SRF08} \\}
\begin{longtable}{lcc}
	\label{tab:Vergleich_SRF05_SRF05}
	\longtableheader
	\endfirsthead
	\longtableheader
	\endhead
	\caption{Vergleich der zwei Ultraschall-Sensoren SRF05 und SRF08}
	\endlastfoot
	\multicolumn{3}{r}{\longtableendfoot} \\
	\endfoot

	Betriebsspannung	& \multicolumn{2}{c}{\unit{5}{\volt}} \\
	Frequenz			& \multicolumn{2}{c}{\unit{40}{\kilo\hertz}} \\
	Reichweite			& \unit{1}{\centi\metre} bis \unit{4}{\metre}
		& \unit{3}{\centi\metre} bis \unit{6}{\metre} \\
	Erfassbare Objekte	& 1					& 16 \\
	Schnittstelle		& Echo-Signal		& \acs{I2C} \\
	Abmessungen			& \multicolumn{2}{c}{\unit{43 x 20 x 17}
		{\milli\metre}} \\
	Besonderheiten		& \textdiscount		& Lichtsensor \\
\end{longtable}

Da diese Sensoren komplexer in der Ansteuerung sind,
benötigte ich noch ein Mikrocontroller Entwicklerboard.
Ich entschied mich für die Open-Source-Prototypen Plattform
Arduino Uno\footnote{\url{\urlArduinoUno}}.
Das Arduino Uno ist mit einem ATmega328\footnote{siehe Datenblatt des Bauteils:
\url{http://www.atmel.com/dyn/resources/prod_documents/doc8161.pdf}}
ausgestattet.

Ich begann damit, indem ich drei Kabel an den SRF05
und vier Kabel an den SRF08, um diese anzusteuern, anlötete.

Dann kam ich zur Programmierung.
Die Programmiersprache, die ich verwendete,
ist eine C ähnliche Hochsprache
(\href{http://wiring.org.co/}{Wiring}).
Aufgrund der Datenblätter und einer Bibliothek,
zur Interaktion mit \ac{I2C}, war die Ansteuerung
der zwei Sensoren relativ einfach zu realisieren.
Ich schloss die Ultraschallsensoren an
die, in der Software definierten Ports, an.
Die Module wurden außerdem mit \unit{5}{\volt} versorgt.
Die gemessenen Werte wurden über eine Serielle Verbindung
an einen Computer übertragen.

Es folgte ein erster Test der Sensoren.
Beide Entfernungsmesser sind, unter optimalen Bedingungen,
auf den Zentimeter genau.
Außerdem detektieren diese Sensoren ein Hindernis auch
wenn es schräger als 20\textdegree\ zu einer Wand steht.
Auch ein kleines Objekt (\unit{5 x 5}{\centi\metre}) wird zuverlässig
aus einer Entfernung von \unit{2}{\metre} erkannt.

\subsubsection{Test der Sensoren unter Realbedingungen}
Nach diesen sehr guten Ergebnissen musste als Nächstes geklärt werden,
ob die SRF-Ultraschallreihe auch im Flug,
unter Luftdruckänderungen durch die Propeller,
zuverlässig funktioniert.
Um dies zu Testen beschaffte ich noch ein \ac{LCD} Display,
um die Messdaten auch ohne Computer im Flug,
ablesen zu können.
Dieses Display lässt sich auch über einen \ac{I2C}
Bus ansteuern.
Um das Display zusammen mit dem SRF08 an den \ac{I2C} anzuschließen
zu können, musste an den Takt- und Datenport jeweils ein Widerstand
von \unit{2,2}{\kilo\ohm} gegen $+$\unit{5}{\volt} angebracht werden.%
\footnote{Die Pull-Up Widerstände sind eigentlich bei jedem \ac{I2C}-Bus erforderlich,
dies viel mir aber erst auf, als ich das Display anschloss~\dots}
Diese kleine Schaltung lötete ich auf eine Lochrasterplatine.

Nun musste noch die Software um die Ansteuerung des \ac{LCD} Display
erweitert werden.
Das Display hat 4 Zeilen mit je 20 Zeichen.
Es erwartet als Ansteuerung eine Bitfolge für ein Zeichen.
So können beliebige Zeichen, die nur durch die Pixelmatrix pro Zeichen begrenzt werden,
angezeigt werden.
Da man aber nicht für jeden Buchstaben die konkrete Darstellung
in Bitfolgen angeben will, existieren sogenannte Codetabellen,
die für jedes Zeichen die entsprechende Bitfolge enthalten.
Diese Codetabelle ist ebenfalls in einer Bibliothek verfügbar.
Somit konnten die Messergebnisse auf dem Display angezeigt werden.
In der Abb. \vref{fig:Ultraschall_Testschaltung}
ist der Testaufbau zu sehen und in Abb. \vref{fig:Ultraschall_Testschaltung_Display}
ist das \acs{LCD} Display dargestellt, inklusive der Anzeige während des ersten Versuchs.

\begin{figure}\centering
%	\includegraphics[width=14cm]{files/images/Ultraschall-Sensoren/Testschaltung//tex/IMG_7317-nocut\imageresize}
	\includegraphics[width=14cm]{files/images/Ultraschall-Sensoren/Testschaltung//tex/IMG_7317\imageresize}
	\captionof{figure}{Ultraschall Testschaltung mit Arduino Uno und \acs{LCD} Display}
	\label{fig:Ultraschall_Testschaltung}
\end{figure}

\begin{figure}\centering
 \LCD{4}{20}|Angaben in cm; 10 Hz|
			|Objekt des SRF05:133|
			|3 Objekte des SRF08 |
			|1:46   2:114  3:243 |
	\captionof{figure}{Das \acs{LCD} Display der Ultraschall Testschaltung}
	\label{fig:Ultraschall_Testschaltung_Display}
\end{figure}

Diese Schaltung befestigte ich provisorisch am Mikrokopter.
Sodass die Sensoren nach unten auf den Boden zeigten.
Dieser Aufbau ist in Abb. \vref{fig:Ultraschall_Testschaltung_Mikrokopter}
zu sehen.

\begin{figure}\centering
%	\hspace{-2cm}		%% Ändern bei Anpassung der Bildgröße
	\includegraphics[width=14cm]{files/images/Mikrokopter-Ultraschall-Test/IMG_7328-rm-bg\imageresize}
	\captionof{figure}{Mikrokopter Testschaltung mit Arduino Uno, Ultraschall und \acs{LCD} Display}
	\label{fig:Ultraschall_Testschaltung_Mikrokopter}
\end{figure}

Die Entfernungsangaben der Sensoren im Flug schwankten durch die Luftverwirbelungen
in \unit{1}{\metre}
etwa um \unit{\pm 10}{\centi\metre}.
In Bodennähe wurden die Werte exakter.

Nun baute ich, auf einer Lochrasterplatine,
eine optimierte, selbst erdachte Schaltung auf,
die den Mikrocontroller, Sensoren und das \ac{LCD} Display direkt verbindet.
Dies ersetzt das Arduino Uno Board.
Die Schaltung ist in Abb. \vref{fig:Ultraschall_Testschaltung_Lochraster} abgebildet.
Zusätzlich lötete ich noch, zu demonstrationszwecken,
10 \acs{LED}s auf die Lochrasterplatine
die die Entfernung des SRF08 in Zehnerschritten (\si{\centi\metre})
anzeigen.
Zusätzlich Programmierte ich noch ein paar Effekte
für die \acs{LED}s die beim Start ausgeführt werden.
So wird zum Beispiel ein Lauflicht dargestellt.
Dieses wird erst immer schneller, bis runter zu \unit{10}{\milli\second}
die zwischen einem Wechsel gewartet werden
und wird dann wieder langsamer.
Programmiertechnisch habe ich zwei For-Schleif benutzt,
die in \unit{10}{\milli\second} Schritten runter
beziehungsweise hochzählen.
In jedem Schleifendurchlauf wird eine Funktion aufgerufen,
die nacheinander die LEDs von 1 bis 10 anschaltet und die vorherige ausschaltet.
Zwischen einem umschalten auf eine benachbarte LED wird die Zeit die von der
For-Schleif an die Funktion übergebenen wurde, gewartet.

Im folgendem ist der Quellcode der zwei For-Schleif abgedruckt.

\lstset{language=C}
\begin{lstlisting}
byte i;
for(i=100; i > 0; i = i - 10){
  KnightRider(i);
}
for(i=10; i <= 90; i = i + 10){
  KnightRider(i);
}
\end{lstlisting}

Die Funktion \enquote{KnightRider} ist etwas länger, da ich mir keine schöne Implementierung
überlegt habe~\dots

\begin{figure}\centering
        \includegraphics[width=14cm]{files/images/Ultraschall-Sensoren/Testschaltung//Lochraster/IMG_7364-rm-bg\imageresize}
        \captionof{figure}{Ultraschall Testschaltung auf einer Lochrasterplatine und \acs{LCD} Display}
        \label{fig:Ultraschall_Testschaltung_Lochraster}
\end{figure}

Nach weiteren Tests mit dem SRF05 und dem SRF08, zeigte sich,
dass dies, für meinen Anwendungsfall, geeignet sind.

\subsection{Herstellen der Verbindung zwischen Mikrokopter und Entfernungssensoren}
Nun folgte der schwierigste Teil meiner Arbeit.
Es war an der Zeit eine Möglichkeit zu finden,
den Mikrokopter dazu zu befähigen,
auf die Sensordaten autonom zu reagieren.

Es gab mehre Möglichkeiten, um dies zu tun.
Zuerst suchte ich eine Schnittstelle zum Mikrokopter,
die es mir erlaubt eigene Steuerbefehle an diesen zu senden
beziehungsweise die Software des Mikrokopter so
anzupassen das dieser selbst die Sensoren abfragen kann.

\subsubsection{Anschluss der Ultraschallsensor an die Hauptplatine}
Hier kam mir die Idee die Ultraschallsensoren direkt an die
Hauptplatine über den \ac{I2C} Bus,
an dem auch die Motorregler angeschlossen sind, anzuschließen.

Ich schloss also den SRF08 an die Hauptplatine an
und modifizierte die Software der Hauptplatine.
Der Quellcode des Mikrokopters ist mit Einschränkung
des \ac{GPS}-Teils verfügbar.

Ich entfernte zu Testzwecken alles aus der
Hauptschleife und fügte Funktionsaufrufe ein,
die den SRF08 über die \ac{I2C} Schnittstelle
ansteuern sollten.
Allerdings reagierte der Sensor nicht auf die Befehle.
Ich vermute das die Hauptplatine ein modifiziertes
\ac{I2C} Protokoll verwendet.
Somit erschien mir diese Möglichkeit als unpassend,
da später über diese Schnittstelle auch noch die Motorregler
angesteuert werden mussten.
Ich rechnete mit Problemen, wenn ich versucht hätte die
Software der Hauptplatine so zu ändern, dass eine Kommunikation mit
dem Ultraschallsensor möglich wäre.\footnote{Änderung 2012: Ich denke das Problem lag an einer
defekten \ac{I2C} Schnittstelle an meiner zweiten FC, die ich zum Testen nutzte.}

Das Anschließen der Sensoren an die \ac{I2C} Schnittstelle der Hauptplatine
ist noch aus einem weiteren Grund problematisch,
da bei einem Kommunikationsfehler auf diesem Bus auch die
Ansteuerung der Motorregler nicht mehr gewährleistet wäre.
Dies würde unweigerlich zum Absturz führen.

An diesem Punkt entschied ich mich gegen diese Möglichkeit.

\subsubsection{Auswertung über einen externen Mikrocontroller}
Die nächste Möglichkeit, die ich versuchte, war einen eigenen
Mikrocontroller mit der Auswertung zu benutzen
und dann die Steuerbefehle an die Hauptplatine zu übermittelt.

Dies ist allerdings um einiges schwieriger als die erste Möglichkeit.
Ich hätte ein (modifizierte) UART Protokoll implementieren müssen.
Dies wird auch zwischen der Hauptplatine zum Computer
oder zum Navigationsboard gesprochen.

Nach einigen Tests gab ich dies Möglichkeit auch auf.

Da dies aber die einzigen zwei praktikablen
Möglichkeiten waren.
Konnte ich mit dem Mikrokopter meine
Zielsetzung nicht umsetzten.

