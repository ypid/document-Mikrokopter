\subsubsection{Magnetfeldsensor}

Der verwendete Magnetfeldsensor (MK3Mag) ist ein 3-Achsen-Magnetfeldsensor.
Somit kann das Erdmagnetfeld im dreidimensionalen Raum erfasst werden.

Die Abmessungen betragen \unit{38 x 22 x 10}{\milli\metre}.

Die Information der Lage des magnetischen Nordpols
kann dann dazu eingesetzt werden, um die Steuerbefehle des Piloten
von der Blickrichtung des Mikrokopters zu entkoppeln.
Das bedeutet das sich die Steuerbefehle nicht mehr auf die aktuelle Blickrichtung,
sondern auf die Blickrichtung, die beim Einschalten erfasst wurde, beziehen.
Somit muss der Pilot nicht mehr in Kenntnis von der aktuellen Blickrichtung
des Mikrokopters sein, um diesen zu steuern.
\begin{wrapfigure}{r}{0pt}
	\includegraphics[width=7cm]{files/images/MK3Mag/oben-1-rm-bg\imageresize}
	\captionof{figure}{Magnetfeldsensor (MK3Mag)}
\end{wrapfigure}
Wenn man beispielsweise das Kommando \enquote{neige dich nach \enquote{Vorne}}
gibt, dann ist \enquote{Vorne} nicht mehr der rote Ausleger sondern
die Himmelsrichtung auf die der rote Ausleger beim Start der Motoren gezeigt hat.

Der Magnetfeldsensor ist auch erforderlich, um eine zuverlässige
Navigation über \ac{GPS} zu ermöglichen.
Da man über \ac{GPS} nur einen Punkt im
dreidimensionalen Raum\footnote{genau genommen in der vierdimensionalen Raumzeit}
erhält.
Daraus geht aber noch nicht die Raum Orientierung des Mikrokopters hervor.
Die Richtung, in die der rote Ausleger zeigt, lässt sich über \ac{GPS} nur erfahren,
wenn sich der Mikrokopter bewegt.
