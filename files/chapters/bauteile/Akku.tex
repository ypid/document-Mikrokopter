\subsubsection{Akku}
\begin{wrapfigure}{r}{0pt}
	\includegraphics[width=8cm]{files/images/Akku/IMG_7099-rm-bg\imageresize}
	\captionof{figure}{Lithium-Polymer-Akku}
\end{wrapfigure}

Der verwendete Akku ist ein im Modellbau
üblicher \ac{LiPo} Akku.
Ein \ac{LiPo} Akku gegenüber \ac{NiCd} oder \ac{NiMH}
hat den Vorteil der höheren Energiedichte
und einer schnelleren Energieauf-
beziehungsweise abgabe.

Die verwendeten Akkus haben, bei 3 in Reihe geschalteten Zellen,
eine Spannung von \unit{11{,}1}{\volt} und eine Kapazität von \unit{2200}{\milli\ampere\hour}.
Die Akkueinheit wiegt \unit{179}{\gram}.
Daraus ergibt sich eine Energiedichte von \unit{136{,}42}{\watt\hour\per\kilogram}.
Die Abmessungen betragen \unit{100 x 34 x 25}{\milli\metre}.

Der Akku wird mit der Verteilerplatine (in meinem Fall mit Klettband) verbunden.
