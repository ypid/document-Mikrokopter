\subsubsection{Brushless-Motor}
\begin{wrapfigure}{r}{0pt}
	\includegraphics[width=7cm]{files/images/Brushless-Mortor/Motor-rm-bg\imageresize}
	\captionof{figure}{Brushless-Motor der Firma Robbe}
	\label{fig:Brushless-Mortor-Robbe}
\end{wrapfigure}

Die Motoren, die sich im MikroKopter Projekt durchgesetzt haben sind bürstenlose Gleichstrommotoren.
Der wichtigste Unterschied zu einem \enquote{normalen} Elektromotor ist,
dass keine Bürsten (Schleifkontakte) benötigt werden,
da der bewegliche Teil des Motors mit Permanentmagneten ausgestattet ist
und auf dem unbeweglichem Teil das Drehfeld erzeugt wird.
Somit ergeben sich folgende Vorteile:
\begin{itemize}
	\item Hoher Wirkungsgrad (da keine Bürsten)
	\item Hohe Lebensdauer
		(wegen dem sehr geringen Verschleiß $\rightarrow$ keine Bürsten)
	\item Gute Regelbarkeit
		(Drehfeld wird extern erzeugt $\rightarrow$ kann gemessen werden)
\end{itemize}

Jeweils ein Brushless-Motor wird von einem Motorregler kontrolliert.
Diese sind über drei Pole miteinander verbunden.

In meiner Konfiguration kommen die Brushless-Motor von Robbe ROXXY
BL-Outrunner 2824-34 zum Einsatz.
% Abb.~\vref{fig:Brushless-Mortor-Robbe}
Diese haben die Abmessungen \unit{28,8 x 26}{\milli\metre}
und wiegen \unit{48}{\gram}.

Auf dem mit Permanentmagneten ausgestattete Teil des Motors (Glocke),
können die Propeller die Befehle der Hauptplatine in Auftrieb umsetzen.
