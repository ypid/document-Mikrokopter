\subsubsection{Modellbau Funkempfänger}
\begin{wrapfigure}{r}{0pt}
	\includegraphics[width=7cm]{files/images/Funkempfaenger/IMG_7119-rm-bg\imageresize}
	\captionof{figure}{Modellbau Funkempfänger}
	\label{fig:Modellbau-Funkempfaenger}
\end{wrapfigure}

Der Modellbau Funkempfänger (ACT DSL-4top) der zum Einsatz kommt,
arbeitet Unverschlüsselt und Analog auf einer Frequenz von \unit{35}{\mega\hertz}.

Das Gewicht beträgt circa \unit{6}{\gram}
bei den Maßen \unit{36 x 13 x 16}{\milli\metre}.

Um die Trägerfrequenz von etwa \unit{35}{\mega\hertz}
zu erzeugen kommt ein Quarz zum Einsatz,
von diesem ist rechts auf der Abb.~\vref{fig:Modellbau-Funkempfaenger}
das Metallgehäuse erkennbar.
Der Sender braucht die gleiche Frequenz um dem Empfänger Signale übermitteln zu könne.
Hierfür wird ebenfalls ein Quarz verwendet.

Die Hauptplatine wird an der ersten dreipoligen Stiftleiste angeschlossen,
da hier das Summensignal anliegt.
Über dieses Summensignal werden alle 9 Kanäle an die Hauptplatine übertragen.

Diese Informationsübertragung geht allerdings nur in eine Richtung (unidirektional).
Sollen Informationen vom Mikrokopter zurück an den Piloten übertragen werden,
kann auf Bluetooth zurückgegriffen werden.
