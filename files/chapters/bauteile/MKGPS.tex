\subsubsection{\acs{GPS}-Empfänger}

Der \ac{GPS}-Empfänger (MKGPS) ist für den Empfänger und die Auswertung des \ac{GPS}-Signals zuständig.
Die Informationen, die sich aus diesem ergeben, werden, dann an das Navigationsboard weitergegeben.
\newpage

\begin{wrapfigure}{r}{0pt}
	\includegraphics[width=6cm]{files/images/MKGPS/unten-rm-bg\imageresize}
	\captionof{figure}{\acs{GPS}-Empfänger (MKGPS)}
\end{wrapfigure}
Die Platine arbeitet im Normalfall mit einer passiven Antenne.
Passive Antenne bedeutet, dass diese nicht über einen eingebauten Signalverstärker verfügt.
Zur Erhöhung der Genauigkeit kann auch eine aktive Antenne benutzt werden,
die aber auch teurer sind.
Die \ac{GPS}-Auflösung die rechnerisch erfasst werden kann,
beträgt \unit{1}{\centi\metre}.
Dieser Wert wird allerdings nicht erreicht,
da die Genauigkeit des, für das US-Militär entwickelten \ac{GPS},
für Zivilisten nicht so hoch ist.
Der Grund dafür ist, dass die Korrekturinformationen von Referenzstationen nicht frei Verfügbar
sind.

Um schneller die Position des Mikrokopters ermitteln zu können,
ist außerdem noch eine wiederaufladbare Knopfzelle auf der Platine aufgelötet.
Diese ermöglicht es den Almanach\footnote{Daten zu den Umlaufbahnen aller \ac{GPS}-Satelliten.}
und die aktuelle Zeit zu halten.
Dies wird getan weil die Datenrate über \ac{GPS} sehr langsam ist (\unit{50}{Bit\per\second})
und deswegen die Übertragung des kompletten Almanachs 12,5 Minuten dauern würde.
%\wrapfill

\acused{USB}
Außerdem befindet sich auf der Platine eine Lötstelle für einen \ac{USB}-Anschluss. Der
\ac{USB}-Anschluss ist direkt mit der \ac{USB}-Schnittstelle vom Empfänger-Modul\footnote{LEA-4H
von \href{http://www.u-blox.com/}{u-blox}.} verbunden.
Dadurch kann man den \ac{GPS}-Empfänger auch vom Computer aus leicht abfragen.
