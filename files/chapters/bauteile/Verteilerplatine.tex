\subsubsection{Verteilerplatine}
\begin{wrapfigure}{r}{0pt}
	\includegraphics[width=9cm]{files/images/Verteilerplatine/IMG_7084-rm-bg\imageresize}
	\captionof{figure}{Stromverteiler-Leiterkarte}
	\label{fig:Stromverteiler-Leiterkarte}
\end{wrapfigure}

Die Verteilerplatine\footnote{Stromverteiler-Leiterkarte (4fach-BL)}
verteilt den Strom vom Akku
und die Steuersignale von der Hauptplatine.
Somit sinkt die Anzahl der benötigten Kabel deutlich.
Auf dieser Verteilerplatine können vier Motorregler eingelötet
und dadurch mit Strom und der \ac{I2C} Schnittstelle verbunden werden.
Auf Abb.~\vref{fig:Stromverteiler-Leiterkarte} sind
die vier Motorregler bereits auf die Verteilerplatine gelötet.
Es ist auch eine Transistorschaltung vorgesehen,
um zwei Verbraucher mit maximal \unit{800}{\milli\ampere}
pro Verbraucher schalten zu können.

Außerdem wird die komplette Stromversorgung des Mikrokopters über einen
elektronischen Leistungsschalter (BTS555), ein Halbleiterrelais, geschaltet.
Da die meisten erhältlichen mechanischen Kippschalter keine hohen Gleichströme
zuverlässig schalten können.
Weil bei Gleichstrom, anders als bei Wechselstrom,
ein viel längerer und anhaltender Lichtbogen entsteht,
der die Kontaktflächen des Schalters schnell abbrennen lässt.
Dadurch steigt sein Innenwiderstand.
Die folge: Hitzeentwicklung und Energieverlust.

Die Verteilerplatine versorgt die Hauptplatine mit Strom
und ist unterhalb der Hauptplatine am Rahmen angebracht.
