\subsubsection{Motorregler}
\begin{wrapfigure}{r}{0pt}
	\includegraphics[width=7cm]{files/images/BL-Ctrl/oben-rm-bg}
	\captionof{figure}{Motorregler (Brushless Regler V1.2)}
\end{wrapfigure}

Der Motorregler ist für die Ansteuerung eines Brushless-Motors zuständig.
Seine Abmessungen betragen \unit{20 x 43}{\milli\metre}.

Er bekommt die jeweiligen Sollwerte von der Hauptplatine
mit einer Frequenz von \unit{500}{\hertz}
und setzt sie in Steuerbefehle für die Motoren um.
Um dies zu erreichen, wird je ein Atmel AVR Mikrocontroller für einen Motor benutzt.

Da für einen Brushless-Motor das Drehfeld\footnote{Ein Drehfeld
ist ein Magnetfeld, das sich um eine Rotationsachse dreht.}
drei Pole mit einer hohen elektrischen Leistung\footnote{Der verwendete Motorregler
ist für eine Dauerbelastung von \SIrange{10}{12}{\ampere}
und eine kurzzeitige Spitzenbelastung von \unit{20}{\ampere} ausgelegt.}
geschaltet werden muss, werden für einen Brushless-Motor drei \ac{MOSFET} als Steuertransistor benutzt.

Die Drehzahlwerte werden in Form eines Drehfeldes an die Motoren weitergeben.
