\subsubsection{Rahmen}
%\begin{minipage}{8cm}
\begin{wrapfigure}{r}{0pt}
	\includegraphics[width=8cm]{files/images/Rahmen/1-rm-bg\imageresize}
	\captionof{figure}{Mikrokopter Rahmen (MK40)}
	\label{fig:MK40}
\end{wrapfigure}
%\end{minipage}

%\begin{minipage}{8.4cm}
Der Rahmen (MK40) bildet das Grundgerüst des Mikrokopter.
Es besteht aus zwei quadratischen Zentralplatten aus \ac{GFK}
und vier Auslegern aus Aluminium Vierkantrohre (\unit{10 x 10}{\milli\metre}).
Ein Ausleger ist rot lackiert und bildet den ersten Ausleger.
Die anderen drei Ausleger sind dagegen schwarz lackiert.
Der rote Ausleger bildet somit den ersten, vorderen Ausleger.
An diesem Orientieren sich die Steuerbefehle des Piloten.

In den Zentralplatten und den Auslegern sind die
Löcher für die Schrauben und Motoren bereits vorgebohrt.
Nach dem Zusammenbau beträgt
der Abstand zweier gegenüberliegender Motoren \unit{40}{\centi\metre}
und das Gewicht des Rahmens wie er auf Abb.~\vref{fig:MK40}
zu sehen ist circa \unit{105}{\gram}.
%\end{minipage}
%\hfill

In der Mitte des Kreuzes wird der Akku und die Elektronik befestigt.
