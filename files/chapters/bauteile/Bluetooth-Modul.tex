\subsubsection{Bluetooth Funkmodul}

Das Bluetooth Funkmodul ermöglicht eine digitale drahtlose Kommunikation mit einem Computer
oder mit einem anderen Mikrokopter.

Das eigentliche Bluetooth Funkmodul (F2M03GXA) arbeitet mit \unit{3{,}3}{\volt}.
Da die Hauptplatine aber mit \unit{5}{\volt} Betriebsspannung arbeitet,
wird noch eine Adapterplatine (BT-AP10) benötigt.
Diese erlaubt den Betrieb dieses Bluetooth Funkmoduls.
Das Bluetooth Funkmodul hat sehr gute Funkübertragungseigenschaften
und ist sehr Energie effizient.
Es hat eine Reichweite bis zu \unit{300}{\meter},
wenn als Gegenstück das gleiche Bluetooth Funkmodul zum Einsatz kommt.
%was eigentlich in keiner Bluetooth Klasse spezifiziert ist
%und somit nur mit dem gleichen Bluetooth Modul als Gegenstelle erreicht werden kann.
Die Bluetooth Verbindung ist hierbei verschlüsselt.

\begin{wrapfigure}{r}{0pt}
\begin{minipage}{7cm}
	\centering
	Adapterplatine~~~~~~~ \\
	\includegraphics[width=7cm]{files/images/BT-AP10/BT-Modul\imageresize} \\
	\vspace{-0.2cm}
	Modul F2M03GXA~~~~
	\captionof{figure}{Bluetooth Funkmodul}
	\label{fig:Bluetooth-Funkmodul}
\end{minipage}
\end{wrapfigure}
Zusätzlich können Störungen auf einzelnen Kanälen erkannt
und diese als unbenutzbar markiert werden.

Auf Abb.~\vref{fig:Bluetooth-Funkmodul} ist in der oberen Hälfte
die selbst gelötete Adapterplatine mit dem auf der Rückseite angelötetem
Bluetooth Modul zu sehen.
Die untere Hälfte zeigt das Bluetooth Modul von der bestückten Seite.

Eine digitale Funkverbindung ist auch notwendig, um dem Mikrokopter im Flug
Zielkoordinaten zu schicken, sodass dies angeflogen werden können.
Die Navigation wird von dem Navigationsboard erledigt
an dem auch das Bluetooth Modul angeschlossen ist.
