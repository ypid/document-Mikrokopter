\subsubsection{Hauptplatine (FlightCtrl)}
\begin{wrapfigure}{r}{0pt}
%\begin{figurewrapper}
	\includegraphics[width=7cm]{files/images/FC/oben-rm-bg\imageresize}
	\captionof{figure}{Hauptplatine (Flight-Ctrl V2.0 ME) von oben}
	\label{fig:FlightCtrl:oben}
%\end{figurewrapper}
\end{wrapfigure}

Die Hauptplatine, in Fachkreisen auch FlightCtrl genannt,
ist die zentrale Steuereinheit des Mikrokopters.
Diese ist zwingend notwendig um die Flugfähigkeit des Mikrokopters zu gewährleisten,
da mit dieser die komplette Lageregelung durchgeführt wird.

Die Hauptplatine hat die Abmessungen \unit{50 x 50}{\milli\metre}
und wiegt (bestückt) \unit{23}{\gram}.

Die Lageregelung geschieht mit drei Beschleunigungssensoren,
die je für eine Raumdimension messen können,
von wo die (Erd)gravitation wirkt.
%\fxnote{Ich möchte zum Ausdruck bringen das nicht nur die Erdgravitation,
%sondern eigentlich nur die stärkste Gravitation die Messergebnisse bestimmt}
Somit ist für die Berechnung die Position der Erde zum Mikrokopter bekannt.
Zusätzlich stehen noch drei
\href{http://de.wikipedia.org/wiki/Drehratensensor}{Drehratensensoren}
zur Erfassung der Drehgeschwindigkeit um die drei Achsen zur Verfügung.
Bewegt sich der Mikrokopter, dann wirkt die Gravitation
und die Eigenbeschleunigung auf die Sensoren
und somit ist die Position der Erde zum Mikrokopter nicht mehr allein
über die Beschleunigungssensoren ermittelbar.
Mit Hilfe dieser zwei Sensortypen und der Software, die vom
MikroKopter Projekt\footnote{Also hauptsächlich von Holger Buss und Ingo Busker} stammt,
kann sich der Mikrokopter eigenständig in einer waagerechten Position halten.
Die Hauptplatine ist hierfür mit den vier Motorreglern
über eine \ac{I2C} Schnittstelle verbunden.

Es steht auch ein barometrischer Luftdrucksensor zur Verfügung,
dessen Wert sich beim Einschalten des Mikrokopters auf die relative Höhe Null eingestellt.
Mit diesem Sensor kann der Mikrokopter seine Höhe selbst erfassen
und dadurch ist es möglich selbstständig eine vorgegebene Höhe halten.

Die dafür notwendige Berechnung werden mit Hilfe eines AVR Atmel Mikrocontroller durchgeführt.
Dieser ist auf Abb.~\vref{fig:FlightCtrl:oben} als schwarzes, quadratisches Halbleiterbauelement
mit vielen (44) Kontakten zu sehen.

Es stehen auch verschiedene Schnittstellen über Steckerleisten zur Verfügung.
Diese dienen dem Verbinden mit einem Computer
oder um weitere Module mit der Hauptplatine verbinden zu können.

Es werden auch direkt Schnittstellen für das Ansteuern von Servos bereitgestellt.
Servos sind im wesentlichen Elektromotoren samt ihrer Steuerelektronik
und Mechanik.
Diese können benutzt werden um, für eine Kamera, die Flugbewegungen
zu kompensieren.
Dadurch ist es möglich immer den gleichen Bildausschnitt aufzunehmen.
Was ohne aktive Kompensation nur mit Gewichten möglich wäre.
Diese Kompensation ist notwendig da sich der
Mikrokopter in die jeweilige Richtung beugen muss
um in diese Richtung zu fliegen.

Um als Pilot den Mikrokopter steuern zu können, ist die Hauptplatine mit einem,
im Modellbau üblichem, Funkempfänger verbunden.
Dieser empfängt die Steuerbefehle des Piloten über eine Antenne.

Die Befehle werden Ausgewertet und in Form von Drehzahlenwerten an die
Motorregler weitergegeben.
