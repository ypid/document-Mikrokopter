\subsubsection{Navigationsboard}
\begin{wrapfigure}{r}{0pt}
	\includegraphics[width=7cm]{files/images/Navi-Ctrl/oben-1-rm-bg\imageresize}
	\captionof{figure}{Navigationsboard (Navi-Ctrl)}
\end{wrapfigure}

Das Navigationsboard (Navi-Ctrl) ist die Navigationsrecheneinheit.
Diese wird benötigt, um die \ac{GPS}- und Kompassdaten zu verrechnen.
Diese Daten werden benötigt um zum Beispiel
Zielkoordinaten anzufliegen.

Die Abmessungen betragen \unit{50 x 50}{\milli\metre}.
Hauptbestandteil des Navigationsboard ist ein ARM9 Mikrocontroller,
ein micro \ac{SD} Kartenleser und mehrere Schnittstellen.

Ist eine Speicherkarte im Kartenleser, dann kann die Firmware des
Navigationsboard in einem gewissen Intervall (Standard: eine Sekunde)
Statusinformationen auf die Karte sichern.
Diese Statusinformationen umfassen beispielsweise die \ac{GPS}-Koordinaten,
die aus dem \ac{GPS}-Signal berechnete
\href{http://de.wikipedia.org/wiki/GPS-Zeit}{Uhrzeit},
die Lage des magnetischen Nordpols und die Akkuspannung.

Außerdem existieren noch mehrere Schnittstellen die zum Verbinden
des Navigationsboard zu der Hauptplatine, dem Magnetfeldsensor
und dem \acs{GPS}-Empfänger dienen.

Das Navigationsboard hat keinerlei Sensoren,
sondern ist lediglich eine notwendige Adapter- und Recheneinheit
um \acs{GPS}- und Kompassdaten auszuwerten.
Die Sensoren werden als Module angeschlossen.
Der Magnetfeldsensor ist direkt auf dem Navigationsboard befestigt.
