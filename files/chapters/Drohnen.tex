\section{Drohnen}
\label{sec:drohnen}

\subsection{Begriffsklärung}
\label{subsec:drohnen:begriffskläaerung}

Eine Drohne ist im klassischen Sinne eine männliche Biene, die zu Tausenden in einem Bienenvolk lebt.
In diesem Volk kommt es nicht auf eine einzelne Biene an, sondern darauf das es genug Arbeiter gibt, um die Arbeiten zu bewerkstelligen.
Aufgrund dieser Entbehrlichkeit der einzelnen Einheiten wurde der Begriff auch für
\ac{UAV} gewählt.

%Für die meisten Militärs dieser Welt ist
Diese Drohnen sind relativ billig, die Ausbildung entsprechender Piloten wäre teurer.
So das diese in ganzen Schwärmen geschickt werden können.

%Die erste Drohne wurde im Zweiten Weltkrieg gebaut und bestand aus einem Sprengkörper und einer
%unförmigen Drehscheibe, die
%mechanisch die Flugkurve kontrollierte.
%% Nachdem ich diesen Text nach 239 Tagen erneut gelesen habe, zweifle ich an der Richtigkeit
%% dieses Satzes

Heute gibt es Drohnen in Flugzeuggröße,
die 72 Stunden in der Luft bleiben können bis hin zum \unit{0{,}06}{\gram} schweren Flugroboter.
Dies wurde erst durch die rasante Entwicklung der Elektronik und Sensortechnik möglich.
%Sie können mithilfe von immer besseren Automatismen und der Entwicklung von \ac{KI} selbst auf ihre Umgebung reagieren und
%sich ihr anpassen.
%\ac{KI} (im Englischen artificial intelligence, AI) ist ein Teilgebiet der Informatik, welches sich mit der
%Automatisierung
%intelligenten Verhaltens befasst.\footcite{wiki:KI}
%Es ist also der Versuch, ein System zu erschaffen, was eingenständig Probleme lösen kann,
%auf die dieses nicht explizit programmiert wurde. Und darüber hinaus\dots

\begin{comment}
\begin{figurewrapper}	%% http://www.spiegel.de/thema/drohnen/
	\includegraphics[width=\hsize]{files/images/Drohnen/image-139405-topicbig-oxxa-spiegel}
	\captionof{figure}[Unbemannte Drohne der US Air Force]{Unbemannte Drohne der US-Luftwaffe (RQ-4 Global Hawk)
		\newline Quelle: \url{http://www.spiegel.de/thema/drohnen/} (abgerufen am 12.\,06.\,2011)}
\end{figurewrapper}
% Ist natürlich Urheberrechtlich geschürzt. Und leider ohne entsprechende Lizenz freigegeben.
\end{comment}

\pagediffValue{subsubsec:drohnen:AbwaegungderArgumente:Einleitung}{sec:drohnen:end}
%% Erzeugt beim ersten LaTeX lauf einen Fehler
\addtocounter{tmp}{-1}	%% Da das Label auf der selben Seite ist, wie dieser Abschnitt
\subsection{Drohnen in der militärischen und zivilen Anwendung}
\label{subsec:drohnen:AbwaegungderArgumente}
Da Drohnen in militärischer Anwendung schon sehr stark in Benutzung sind
und dieser Trend nicht aufhaltbar scheint,
beschäftige ich mich auf den nächsten
\fnumprintc{tmp} Seiten mit den Fragen,
die mit der militärischen und zivilen Anwendung von Drohnen,
aufgeworfen werden.

\subsubsection{Einleitung}
\label{subsubsec:drohnen:AbwaegungderArgumente:Einleitung}
Die Entwicklung von autonomen Robotern wird maßgeblich
durch den militärischen Sektor vorangetrieben.
Hauptsächlich um die Kriegsführung noch effizienter zu gestalten.
Aber es gibt auch viele zivile Projekte,
die friedlichere Ziele verfolgen.
Die technische Entwicklung verläuft um einiges schneller, als die
Fragen die durch die Benutzung solcher unbemannter autonomer Systeme
auftreten, beantwortet werden können.
Solche Fragen haben häufig einen ethischen, politischen
und juristischen Hintergrund.
Konkret steht auch die Frage zur Debatte, ob nicht der Einsatz dieser neuen technischen
Möglichkeiten eine erhöhte Abstraktion zu dem eigentlichen Geschehen zur Folge hat
und somit die Hemmschwelle der Entscheidungsträger in der Kriegsführung sinkt.

Der Einsatz von Drohnen bietet ohne Zweifel viele Vorteile, aber auch Nachteile.
Ich werde versuchen im nachfolgenden Text beide Seiten näher zu beleuchten.
Aufhänger dieses Textes ist der Artikel von \textcite{sdw:2010-12:DfK}.

\subsubsection{Drohnen im Zivilen}
\label{subsubsec:drohnen:AbwaegungderArgumente:Zivilen}
Betrachten wir zuerst den Einsatz von Drohnen im zivilen Sektor.

Drohnen werden immer billiger, somit werden diese immer mehr
zu einer Freizeitgestaltung.
Eine Drohne für den zivilen Gebrauch basiert häufig auf dem Funktionsprinzip
eines Quadrokopters.

Ein Quadrokopter verfügt über vier oder mehr Propeller die kreuzförmig angeordnet sind.
Zwei gegenüberliegende Propeller drehen sich im Uhrzeigersinn
und die anderen zwei drehen sich gegen den Uhrzeigersinn.

Aufgrund dieses Funktionsprinzips werden keine verstellbaren Propeller benötigt
und es wird auch keine Energie für das Ausgleichen des Drehmomentes in der Hochachse benötigt,
wie dies bei einem Hubschrauber mit dem Heckrotor der Fall ist.

Die Manövrierbarkeit eines Quadrokopters ist dennoch gegeben,
da durch Veränderung der Drehzahl ein Kippen des Quadrokopters erreicht werden kann.
Dies geschieht, indem ein Propeller um den gleichen Betrag schneller wird, wie der zweite
gleichläufige Propeller langsamer ist.
Ein Kippen nach Vorne oder Hinten wird als Nicken
und ein seitliches Kippen als Rollen bezeichnet.

Außerdem kann auch noch eine Drehung um die Hochachse ausgeführt werden.
Indem die Drehzahl zweier gleichläufiger Propeller verringert
und dieser Auftriebsverlust durch die zwei anderen Propeller kompensiert wird.
Dies wird als Gieren bezeichnet.

Ein Selbstbauprojekt, das dieses Prinzip umsetzt ist der von
Ingo Busker und Holger Buss initialisierte, \href{http://www.mikrokopter.de}{MikroKopter}.
Es gibt zudem noch das kommerzielle Projekt \href{http://www.microdrones.com}{microdrones}
und inzwischen auch noch eine Reihe weiterer Projekte.

\bigskip
Der einfachste Fall ist das manuelle fliegen der Drohne, wie im Modellbau üblich.
Man kann aber auch mit Hilfe einer Videobrille fliegen,
also nach den Videobildern von der Drohne, die direkt auf zwei
kleine Bildschirme übertragen werden. So hat man das Gefühl selbst in
der Drohne zu sitzen, aber ohne die Gefahr bei einem Absturz Schaden zu
nehmen.

Man kann auch versuchen mithilfe einer Drohne verschiedene Aufgaben zu lösen,
wie zum Beispiel Lasten mit einem an der Drohne befestigtem Haken aufnehmen
und wieder absetzen oder das Umfliegen von Hindernissen.
Oder man fliegt Figuren (Kunstflug) oder ohne bestimmte Vorsätze.

%Man kann eine Drohne natürlich auch etwas weniger spielerisch einsetzen.
%Beispielsweise in dem man, auf seinem Grundstück eine Drohne
%startbereit hat, um im Bedarfsfall eine variable Überwachung zur
%Verfügung zu haben. Um sich einen Überblick verschaffen zu können.
%Somit lässt sich die Anzahl nötigen Überwachungskameras, um ein Gelände
%zu überwachen, stark reduzieren und damit kosten sparen, um dieselbe
%Abdeckung eines Bereiches zu ermöglichen.

Außerdem lassen sich Bilder und Videos aus der Vogelperspektive
aufnehmen, was schon von vielen Berufsgruppen genutzt wird.
Beispielsweise bei Unfällen und Katastrophen um die Rettungskräfte besser zu koordinieren
und eine Dokumentation des Geschehens zu ermöglichen.

\bigskip

Der Vorteil einer Drohne ist, der im Verhältnis zu einem Modellhubschrauber
oder gar einem kleinen Flugzeug niedrigere Preis,
den eine Drohne im Eigenbau kostet.
%dass über die gleichen Fähigkeiten verfügt
Außerdem ist bei einem üblichen Quadrokopter die Wahrscheinlichkeit klein,
dass ein ernsthafter Schaden durch menschliches oder technisches Versagen entsteht.
Da durch Techniken wie \ac{GPS}, Höhenregelung,
Kompass und Kollisionsvermeidung der Pilot bei der Steuerung unterstützt wird,
und im Idealfall, Fehler nicht ausgeführt werden.
Außerdem bietet eine Drohne Techniken wie \ac{GPS}
die neu im Modellbaubereich sind.
Wie zum Beispiel das Automatische zurückfliegen zur Startposition
oder das Halten der Position.
Da aber Schäden nie ausgeschlossen sind, gibt es im privaten
Bereich Modellbau-Haftpflichtversicherungen,
die eventuelle Personen- und Sachschäden übernehmen.

Ein Nachteil hingegen ist die mögliche Verletzung der
Persönlichkeitsrechte eines Nachbarn.
Das Persönlichkeitsrecht einer Person ist beispielsweise dann verletzt,
wenn diese gezielt gefilmt wird.
Oder auch wenn man in dessen Luftraum fliegt, was mit Geräuschen verbunden ist
oder das Filmen des Grundstücks ohne dessen Einwilligung.

\subsubsection{Drohnen in militärischer Anwendung}
\label{subsubsec:drohnen:AbwaegungderArgumente:Militaer}

Im militärischen Bereich werden Drohnen und andere technische
Hilfsmittel immer wichtiger.
So sind bei der US-Armee circa \numprint{7000} \enquote{unbemannte} Fluggeräte
und \numprint{12000} Bodenfahrzeuge im Einsatz.

Im militärischen Bereich wird es darauf hinauslaufen,
dass der Mensch der Maschine nur noch Aufträge überträgt,
die diese ausführen sollen.
Der Roboter kann dann selbstständig in einem Einsatz entscheiden.
Eine solche Entscheidung könnte zum Beispiel die
Tötung einer Person betreffen.

Da aber Roboter, anders als Menschen, einen Auftrag unter allen
Umständen ausführen, werden selbst fehlerhafte Befehle ausgeführt.
Ein Mensch würde diese Fehler erkennen und abbrechen.
Was ein Nachteil des Roboters ist, da hierdurch
Soldaten oder auch Zivilisten verletzt oder getötet werden.
%Dieses Problem wird sich wohl zukünftig durch bessere Sensoren im
%Zusammenspiel mit verbesserter Software kontrollieren lassen.

Der Vorteil ist, dass Roboter militärisch präziser sind, also eine
höhere Trefferwahrscheinlichkeit als Menschen haben.
Zudem sind Drohnen billiger.
Somit wird der Einsatz von Drohnen und Robotern im Krieg
bis hin zum vollständig geführten Maschinenkrieg zunehmen, was
die Schwelle solch einen Krieg zu beginnen, senkt.
Da es für die Befehlshaber mit weniger Problemen
und Diskussionen verbunden sein wird, Roboter zu Schicken,
da die eigenen Soldaten keiner Gefahr mehr ausgesetzt sind.

\bigskip

Hier zeichnet sich aber direkt das nächste Problem ab.
Denn der Mensch, der einen Roboter fernsteuert,
kann Tausende von Kilometern vom Geschehen entfernt vor einem Bildschirm setzen
und wird daher die Situation anders einschätzen, als wenn er vor Ort wäre.
Es ist für diese Person nicht leicht, zu erkennen ob das, was sich da auf
dem Bildschirm abspielt, real oder virtuell ist.
Es könnte dazu übergehen, dass es sehr schwer wird, zu unterscheiden
ob das, was man auf dem Bildschirm abgebildet sieht virtuell ist
oder an einem andern Ort wirklich in die Tat umgesetzt wird.
In solch einem Fall wäre es denkbar 15jährige Jugendliche vor den Bildschirm zu setzten,
da diese, im Vergleich zu anderen Altersgruppen,
eine sehr niedrige Reaktionszeit besitzen und somit effizienter sind.
Diese würden die Tatsache, dass sie auf reale Menschen schießen wohl mit der Zeit verdrängen
oder gar nicht erst erfahren, was ihr Handeln wirklich bewirkt.
Für diese wäre es eine \enquote{schöne} Freizeitbeschäftigung,
was sie sonst wohl auch nicht anders tuen würden,
nur hier erhalten sie noch Geld dafür~\dots

Der nächste denkbare Schritt ist ein
Algorithmus\footnote{Freund-Feind-Erkennung nach dem Motto
\enquote{Erst schießen, dann fragen stellen}},
der auf alles schießt, was als Mensch erkannt wurde und eine Bedrohung darstellen könnte.
Ein Problem ist hier, dass ein Roboter mit Hilfe eines Algorithmus zu einer
zuverlässigen Freund-Feind-Erkennung in der Lage sein muss,
um in mehr Bereichen eingesetzt werden zu können.

Diese Szenarien sind durchaus real und keine ferne Zukunftsvision. Um
dem entgegenzuwirken, formulierte der Science-Fiction-Schriftsteller
Isaac Asimov schon Mitte des 20ten Jahrhunderts die drei aufeinander
aufbauenden asimovschen Gesetze.\footcite{Robotergesetze:Asimov}

\begin{enumerate}
	\item Ein Roboter darf kein menschliches Wesen verletzen oder durch Untätigkeit gestatten,
		dass einem menschlichen Wesen Schaden zugefügt wird.
		\label{Robotergesetze:Asimov:Eins}
	\item Ein Roboter muss den ihm von einem Menschen gegebenen Befehlen gehorchen --
		es sei denn, ein solcher Befehl würde mit Regel
		\hyperref[Robotergesetze:Asimov:Eins]{eins} kollidieren.
		\label{Robotergesetze:Asimov:Zwei}
	\item Ein Roboter muss seine Existenz beschützen, solange dieser Schutz nicht mit Regel
		\hyperref[Robotergesetze:Asimov:Eins]{eins} oder
		\hyperref[Robotergesetze:Asimov:Zwei]{zwei} kollidiert.
\end{enumerate}

Diese haben aber keine Relevanz im Militärischen,
was von Friedensaktivisten nicht ganz ohne Widerstand akzeptiert wird.

Anfang 2008 forderten britische Friedensaktivisten das
autonome Roboter nicht die Entscheidung treffen dürfen,
einen Menschen zu töten.
Das gleiche Problem besteht mit der neuen Generation von
Streubomben. Diese suchen mit Infrarotsensoren den Boden nach
Hitzequellen ab. Werden keine entdeckt, explodieren sie noch in der
Luft, um nach dem Konflikt keine Menschen zu gefährden. Ob dieses
Verbot ähnlich erfolgreich wie das
\href{http://www.un.org/millennium/law/xxvi-22.htm}{Ottawa-Abkommen}
von 1997, das Antipersonenminen verbietet und mittlerweile von über
150 Ländern ratifiziert (umgesetzt) wurde, wird, bleibt
abzuwarten.\footcite{heise:VerbotvonKampfrobotern}

\bigskip

Es bleibt auch noch die Frage was passiert,
wenn jemand durch eine Maschine verletzt oder getötet wird.
Auch das ist schon vorgekommen und die
Hersteller oder Besitzer solcher Maschinen haben sich meist mit
Schadensersatzzahlungen aus dem Fall befreit.
Aber wer wirklich schuldig ist, bleibt schwer zu klären.
Da eine Maschine nicht über einen eigenen Willen oder Bewusstsein verfügt
und nicht als Mensch angesehen wird.
Es muss also nicht von einem vorsätzlichen,
beziehungsweise beabsichtigtem Verbrechen der Maschine ausgegangen werden.
Die Maschine ist also momentan nur ein Werkzeug und kann nicht selbst eine Tat verüben,
sondern nur dazu gebaut werden.
Wie beispielsweise eine Pistole die nicht selbst einen Mord begeht,
sondern erst durch einen Menschen zur todbringenden Waffe wird.
Unsere Gesetze, die beispielsweise den Tatbestand des Mordes regeln,
lassen sich also nicht auf Maschinen übertragen.

Es bleibt noch die Möglichkeit den Hersteller in die Verantwortung zu
ziehen.
Dazu müsste sich aber beweisen lassen, dass eine nach dem jeweiligem Gesetz
strafbare Handlung vom Hersteller vorgesehen war oder Fahrlässigkeit vorliegt.
Da Fahrlässigkeit durch einen Programmier- oder Konstruktionsfehler hier
nicht einfach nur zu einem Programmabsturz führt,
sondern im Zweifelsfall werden Menschen getötet.

Der Anwender oder Besitzer würde sich ebenfalls rechtfertigen müssen.
In diesem Fall wird wieder, ähnlich wie bei dem Hersteller,
die Absicht des Anwenders oder Besitzer geklärt werden müssen.

Wenn es jedoch zu einem wirklichen Bewusstsein, in der Maschine,
kommt, ändert sich wohl einiges.

In der Robotik könnte es dazu übergehen eine \ac{KI} selbst dafür zu
bestrafen, dass diese, ähnlich wie ein Mensch aus den eigenen Fehlern
lernen kann.
Dies wird unter anderem in dem Film
\enquote{\href{http://www.imdb.de/title/tt0343818/}{iRobot}} so skizziert.
In dem ein Roboter beginnt einen eigenen, freien Willen zu entwickeln
und somit nicht mehr an die asimovschen Gesetze gebunden ist.
Dieser begeht einen Mord und wird verhaftet.
Die Anklage ist jedoch schwer möglich, da der Roboter nicht als Mensch
angesehen wir und somit keinen Mord begehen kann.
Auf solch ein Zukunftsszenario ist noch keiner vorbereitet.
%Das Wort Mord müsste neu definiert werden~\dots

\bigskip
Im folgendem möchte ich noch einen Ausblick auf Roboter,
die sehr viel kleiner sind, wagen.
Wobei es sich allerdings noch größtenteils um Fiktion handelt.

Die Roboter dieser Größe werden als Nanoroboter,
beziehungsweise Nanobots bezeichnet.
Diese Roboter haben eine Größe die einem Blutkörperchen entspricht.
Diese haben sicher ein enormes Potenzial in der Medizin, da sie sich im menschlichen
Organismus, zum Beispiel im Blutkreislauf, bewegen können und dadurch an so
gut wie jede Stelle im Körper gelangen können.
So können diese den Organismus in der Erhaltung der Körperfunktionen unterstützen.

Es ist denkbar dass Nanobots auch zum Recycling von \enquote{Müll} eingesetzt werden.
Da diese Nanobots Atome und Moleküle auf atomarer Ebene neu
anordnen und so benötigte Rohstoffe aus anderen Rohstoffen erzeugen
können oder direkt zu einem fertigen Objekt zusammenbauen können.
%Zudem können sich Nanobots selbst replizierenden
Zum Beispiel wen jemand in der Zukunft einen neuen Esstisch benötigt,
geht er am besten zum Designer seines Vertrauens
und lässt sich den Tisch am Computer entwerfen.
Der Designer schickt den Datensatz dann an eine Werkstatt,
in der der Tisch in einem einzigen Arbeitsgang Atom für Atom zusammengesetzt wird.
Ohne Fugen, Schrauben oder Dübel.
Der Tisch könnte dabei so robust wie Stahl, aber leicht wie Kunststoff sein.
Die Atome wären in einer Weise angeordnet, die an keinen bekannten Rohstoff erinnert.
\footcite{Drexler:EnginesofCreation}

Zugleich geht aber auch eine hohe Gefahr von diesen Nanobots aus.
Zum Beispiel als Spionagemöglichkeit, indem sich die Nanobots in der Umgebungsluft
als \enquote{intelligenter Staub} befinden und so eine umfassende Überwachung
eines Gebietes ermöglichen.

Eine Erkennung dieser Nanobots dürfte sich ohne moderne Hilfsmittel sehr schwer gestalten,
geschweige denn die Abwehr oder Beseitigung,
da man zur Erkennung, dieser winzigen Strukturen, wohl ein Elektronenmikroskop bräuchte.

Wobei aber die Meinungen über den Nutzen, den uns die Nanobots tatsächlich bringen
werden, sehr auseinander gehen,
da die im vorherigen Abschnitt beschriebenen Möglichkeiten
noch theoretischer Natur sind und noch nicht praktisch umgesetzt werden konnten.

%\newpage
\subsubsection{Resümee}
\label{subsubsec:drohnen:AbwaegungderArgumente:Resuemee}

Die militärische Anwendung von Drohnen
halte ich für nicht sonderlich erstrebenswert
für unsere Gesellschaft, da dadurch ein noch höherer Grad an Sicherheit
geschaffen werden soll.
% gesundes Maß an Sicherheit ist zu suchen
Beispielsweise durch zusätzliche und permanente Überwachung.
Was nicht unbedingt mein Bestreben ist.
Da dies zu einem höheren Maß an Kontrolle führt,
wodurch jegliche Freiheit und Individualität verloren gehen könnte.
Wodurch auch die Kreativität der Menschen abnehmen wird.

Im zivilen Sektor sehe ich solche Gefahren weniger.
Da die Entwicklung und Benutzung von Drohnen im Zivilen
in der Regel mit anderen Zielen betrieben wird.
Also nicht die gezielte Tötung anderer Menschen,
was das Ziel bei der militärischen Anwendung ist.
Aber auch im zivilen Sektor werden Drohnen zunehmend zur
Überwachung (von öffentlichen Plätzen) eingesetzt,
was mit den gleichen Gefahren verbunden ist,
wie bei der militärischen Nutzung.

\bigskip
Ich betrachte allgemein die technische Entwicklung als Weiterführung
der biologischen Evolution,
aufgrund der rasanten Entwicklung in der Informatik,
deshalb werden eines Tages Maschinen uns
Menschen, als die \enquote{Krone der \enquote{Schöpfung}}
beziehungsweise der am höchsten Entwickelten Lebensform
auf dieser Erde, ersetzen können.

Zu diesem Schluss komme ich aufgrund der sehr beeindruckenden Fortschritte
in der Technik und im Bereich \ac{KI} in den vergangenen hundert Jahren.
Als einfaches Beispiel:
Das menschliche Gehirn kann Schätzungen zufolge 2 Petabyte
an Informationen in chemischer Form in den Synapsen halten.
%% http://de.wikipedia.org/wiki/Gehirn#Speicher
%% http://www.wolframalpha.com/input/?i=2PB
Diese Kapazität wird auch schon von einem Rechenzentrum erreicht.%
\footnote{Mir ist natürlich klar, dass das Gehirn weit mehr ist als ein großer Datenspeicher~\dots}

Stehphen Howking sagte zur Entwicklung von \ac{KI} folgendes: \\
\enquote{Einige Menschen behaupten Computer würden niemals in der Lage sein,
echte Intelligenz zu entwickeln, was auch immer das sein mag. Doch
wenn komplizierte chemische Moleküle im Menschen so zusammenwirken
können, dass sie diesen mit Intelligenz ausstatten, dann sehe ich
nicht ein, was ebenso komplizierte elektronische Schaltkreise daran
hindern sollte, Computer zu intelligentem Verhalten zu befähigen.}
\footcite{Hawking:Zitat:KI}

Auch wenn es sich bei seiner Aussage um einen Zirkelschluss handelt,
da er mit der unbewiesenen Aussage beginnt, dass
\enquote{komplizierte chemische Moleküle im Menschen so
zusammenwirken können, dass sie diesen mit Intelligenz ausstatten}
so ist seine Folgerung dadurch auch nicht widerlegt und kann
dennoch eintreten~\dots

%Bis es aber soweit ist, dass Maschinen über echte Intelligenz
%verfügen, sollten wir darauf achten, dass dies nicht ins Negative
%umschlägt.

%\bigskip
%Im militärischen Sektor sollte es aber meiner Meinung nach zu stärkeren
%Kontrollen und Einschränkungen, bezüglich des ferngesteuerten Töten,
%kommen da hier ohne Gerichtsverfahren auf Verdacht die Todesstrafe
%angewendet wird beziehungsweise es sollte an die Umsetzung der
%asimovschen Gesetze gedacht werden, auch wenn dies die Nutzung von
%Robotern und Drohnen für die Tötung von Personen untersagen würde.

%Die Umsetzung dieser Forderungen liegt bei der Politik, der, in diesem
%Gebiet, führenden Ländern.

\bigskip
Die weitere Entwicklung der Roboter bleibt also mit Vorsicht zu behandeln,
denn \enquote{letztlich wissen wir aber trotz aller Vermutungen noch nicht,
wie sich die Robotik entwickeln und wohin sie uns führen wird.}\footcite[74]{sdw:2010-12:DfK}
\label{sec:drohnen:end}
